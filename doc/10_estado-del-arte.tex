\section{Estado del Arte}
\label{sec:estadodelarte}

\subsubsection{Estudio sobre Sinestesia}

Dentro de la sinestesia, la parte que es importante en este proyecto es la llamada sinestesia musical. Aunque no es el objetivo realizar una investigación rigurosa y llegar a una conclusión o afirmar una suposición realizada con antelación, estudiamos la sinestesia porque cabe pensar que sería una aproximación muy buena para poder empezar a realizar la conversión de imagen a música.
Según \emph{(Video Redes:Flipar en Colores)}David Eagleman las personas de forma innata  tenemos la tendencia de conectar sentidos, entre otros visual y auditivo (sonidos con formas y sonidos con colores).\\

Varios artistas en el tiempo han expresado su capacidad sinestésica a través de sus obras o documentando y experimentando con este fenómeno. Entre ellos se encuentra Scriabin, quien hizo una asociación entre los colores y los acordes que aparecen en música de forma rigurosa. También hay científicos como Newton que intentaron asignar colores a la notas pero no es un problema reciente, ya desde la antigua Grecia con Aristóteles y Pitágoras se lleva intentando dar una solución definitiva. Después de ver la aproximación a través de la experiencia o a través de la lógica o fundamentación matemática queda claro que cada persona sinestésica es única y puede haber grandes diferencias entre unas experiencias sinestésicas y otras.\\

Las investigaciones llevadas a cabo por el psicólogo Wolfgang Köhler en 1929 y las hechas recientemente demuestran que tenemos una conexión profunda entre los sonidos y las formas de lo que vemos. 

\todo{creo mejor no hablar de esto:Su experimento se basaba en dejar asignar dos palabras (sonidos) a dos figuras...kiki-booma}

 Es por tanto importante poder encontrar esa asociación y ponerla en práctica a la hora de convertir las imágenes en música. Lawrence E. Marks, The Unity of the Senses encuentra una asociación entre los tiempos en música y las formas, que cuanto más angular e irregular es una figura el tempo o las notas son más rápidas.También declara sobre los colores que \emph{``Sadly, one finds that synesthetic associations between colors and musical notes fail to favor any particular scheme over the others.''}: Por desgracia, al final uno descubre que no hay una asociación sinestésica entre notas musicales y colores que prevalezca ante las demás.\\

Kandinsky, otro artista como Scriabin pero partiendo de la pintura, ha investigado y defendido fervientemente la posibilidad de asociar la música a la pintura.De varios de sus estudios y siguiendo sus propias experiencias sinestésicas ha escrito abundante documentación. Relaciona otros aspectos como por ejemplo que la escala de intensidad de los sonidos estaba relacionada con la intensidad de los trazos de la pintura, también distingue timbres de instrumentos por colores asignando a cada familia colores parecidos. \\


\todo{ampliar la sección}

\subsection{Composición basada en imagenes}

\todo{ Esquema de la sección:

\begin{itemize}
\item Durante la historia los experimentos de color-música
\item Técnicas de composición más actuales
\item Pequeña conclusión o coletilla...
\end{itemize}

}

Para poder componer música a partir de datos gráficos necesitamos crear una correlación entre los elementos que tenemos en la parte gráfica y en la parte musical. Esta correspondencia no es fácil de conseguir, y a través de la historia se han desarrollado varias teorías que se han puesto en práctica. Los primeros intentos de síntesis fueron los instrumentos llamados ``órganos de color'' que mostraban acompañando al sonido una muestra visual.

El primer instrumento que se construyó fue por parte de Louis Bertrand Castel (1730) y se trataba de un clavecín al que le incorporaron una pantalla encima y un sistema de iluminación. Cuando se apretaban las teclas se iluminaban los colores correspondientes en la pantalla (\cite{organosColor}). A este experimento le continuaron muchos otros, pero al ser una correspondencia muy directa entre nota y color da poca diversidad, además es una correspondencia de música a visual.\\

Desde hace unas décadas, gracias a los ordenadores, se ha investigado activamente en algoritmos automáticos de composición. Una posible clasificación de los diferentes algoritmos basada en su característica principal sería: modelos matemáticos, sistemas basados en conocimiento, gramáticas, evolutivos, sistemas con aprendizaje e híbridos (\cite{AIMethodsForComposition}). 
En los matemáticos destacamos Stochastic (Markov chains) y Chaos, su principal inconveniente es representar a alto nivel la música. En los sistemas basados en conocimiento como CHORAL o el armonizador de Pachet and Roy tienen la dificultad de representar todo el conocimiento que en este dominio es muy grande. En las gramáticas se tiene como ejemplo el proyecto EMI (Experiments in Musical Intelligence) o Steedman y su generador de música Jazz. El mayor problema que presentan es la rigidez de las propias gramásticas ya que la música es ambigua y con muchas excepciones a las reglas. Los evolutivos su principal problema es la función de evaluación o fitness, tenemos McIntyre que usa una función de evaluación automática y la improvisación novicia de jazz de Biles que usa una evaluación humana. En los sistemas con aprendizaje su mayor inconveniente es que estos sistemas aprenden solo de la base de la música compuesta y no de los niveles más altos de composición, destacamos EBM y MUSE. Los sistemas híbridos intentas combinar lo mejor de los diferentes sistemas posibles, dentro de esta clase encontramos HARMONET que combina sistemas con aprendizaje y basados en conocimiento.

Pero no solo la algoritmia es variada, también podemos clasificar de diferente forma según el tipo de música que se quiera componer: micro-composición (diseño de sonidos) y la macro-composición (combinación de sonidos ya diseñados) (\cite{AudioVisualSurvey}). En nuestro caso queremos separar los algoritmos de macro-composición en: basados en el fenómeno de la sinestesia y los que no siguen esta línea.\\

Hay varias aproximaciones de compositores basados en imagen , algunos de ellos como Phonogramme (\cite{ImageBaseComposition} \cite{Phonogramme}) que es un editor gráfico, interpreta la imagen como una partitura. La imagen representa la relación bidimensional de tonos y duración de los sonidos. El problema de este algoritmo desde el punto de vista de la sinestesia es que las imágenes son partituras y deben estar diseñadas para ser usadas para este fin, no acepta cualquier entrada gráfica. Esta línea de investigación no sigue o intenta imitar la sinestesia.\\

Otra opción interesante es la propuesta basada en el concepto de ``croma'' (o índice de cromatismo). Se subdivide la imagen en bloques (ladrillos cromáticos) los cuales tienen un índice de cromatismo y sirve para generar o asignar trozos de música (\cite{bricksConvertsMusic}). Estos trozos de música pueden cogerse de una base de datos o ser creados por un experto (compositor). Al no haber una correspondencia directa entre la imagen y la música, este proyecto tampoco sigue la línea de la sinestesia.\\

También destacamos el trabajo realizado por Xiaoying Wu y Ze-Nian Li (\cite{ImageBaseComposition}) en el que analizan la imagen en tres pasos. Primero \emph{Hacer una partición} de la imagen en piezas (divisiones, trozos) más pequeñas. Segundo realizan la \emph{Secuenciación} de esas piezas para darles un orden en el tiempo. Por último aplican un \emph{Mapeado} de las piezas de imagen en notas musicales. Al ser una correspondencia directa entre la imagen y la música, este trabajo sigue parcialmente la sinestesia y la psicología de las formas y colores aunque no es su propósito central.\\

Por último destacamos el trabajo de A. Pintado (\cite{portutesis}) en el que hace una investigación de la sinestesia y la percepción de las formas para poder generar ritmos. Analizando una imagen de entrada sacando sus formas y líneas hace una relación directa con diferentes ritmos que genera. Este trabajo se centra en la línea de la sinestesia y por tanto ha sido una fuente de información importante.

Todos estos algoritmos tienen en común un problema, que está implícito, y es que dada una imagen cada persona espera una música diferente como correspondencia. Por tanto es difícil determinar el fitness o validez de cada algoritmo sabiendo además que la entrada gráfica puede tener infinidad de valores posibles.

