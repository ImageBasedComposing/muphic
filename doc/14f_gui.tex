\subsubsection{Interfaz Gráfica}

El desarrollo de la interfáz gráfica se decidió realizar a través de un framework que facilitase la construcción de la misma. Qt, la herramienta que se utilizó se eligió debido a dos criterios, era multiplataforma, al igual que la aplicación, servía para trabajar con móviles, siguiendo así uno de los enfoques iniciales del proyecto que más adelante fue descartado. Además Qt trabaja con C++ como lenguaje de programación, el mismo que el núcleo de la aplicación y ofrece facilidades a la hora de integrar audio, a través de la librería Phonon, o componentes personalizados en una interfaz.\\
\todo{Este párrafo no se si abriría mejor la parte de arquitectura}
\newline
La interfáz esta compuesta de tres pestañas cada una encargada de una parte de la funcionalidad:
\newline
\\\underline{Main Window:}
\\Esta pestaña se encarga de la interacción directa con el usuario, muestra los resultados obtenidos y permite lanzar los distintos componentes de la aplicación. 
\\Esta compuesta por los siguientes elementos tal y como se puede ver en la imagen adjunta:
\\\todo{Adjuntar imagen}
\newline
\\\textit{Barras de exploracion:} Son dos barras en las que el usuario puede elegir las direcciones en las que se cargará la imagen a analizar y se guardarán los archivos de audio. Para elegir la dirección podrá insertar el texto a mano con el nombre que desee o utilizar los botones para usar el explorador del sistema operativo correspondiente.
\\\textit{Input Image y Analysis output:} En el primero de estos dos componentes se muestra la imagen que se va a analizar, en el segundo se mostrará el resultado del análisis una vez lanzado Phic a ejecución.
\\\textit{Botones de Analyze y Compose:} Tal y como su nombre indica Analyze realiza el analisis de la imagen pasada como parámetro de entrada lanzando a ejecución el programa Phic, uno de los módulos de la aplicación. Compose realiza la composición musical a partir de los datos analizados, para ello lanza a ejecución el programa Mu, otro de los módulos de la aplicación.
\\\textit{Botones del reproductor:} Estas erramientas estan comunicadas con la librería Phonon que se encarga de la interacción con el archivo de audio creado, y una vez compuesta la melodía tomarán el parametro de entrada como referencia y cuando el usuario pulse los distintos botones realizara las distintas funciones de play, pause, stop, cambiar el volumen, o moverse a lo largo de la melodía.
\newline
\\\underline{Graphic Config:}
\\En esta pestaña se encuentran todas las opciones relacionadas con el análisis de la imagen, las distintas configuraciones se transmitirán a los módulos de la aplicación a través de un documento XML que se genera cuando en la "Main Window" se le da al boton de "Analyze".
\\Los distintos componentes de esta pestaña van variando según el filtro que se seleccione, sin embargo algunos de ellos son comunes para todos los filtros:
\newline
\\\textit{Filter:} Este combo box permite seleccionar entre los distintos filtros que se pueden usar en la aplicación.
\\\textit{Noise Selection:} Con esta barra de desplazamiento se podrá elegir a partir de que tamaño, relativo al area del mayor polígono de la imagen, se ignorarán los poligonos encontrados durante el análisis de la imagen.
\\\textit{Analysis Depth:} Esta barra permite seleccionar cuanto deseamos comprimir la imagen original antes de analizarla, cuando menor sea su valor menor será el nivel de detalle obtenido y más rápido se realizará el análisis.
\\\textit{Polygon Simplification:} Esta barra permite seleccionar cuanto deseamos simplificar los poligonos obtenidos a partir de la imagen original, cuanto mayor sea su valor menos vertices tendrán los poligonos obtenidos y por lo tanto menos fiel será el resultado del análisis, sin embargo, este se realizará más rápido.
\newline
\\Según el filtro que se elija aparecerán mas componentes:
\newline
\\\todo{Rellenar esto de manera experta y didáctica}
\\\textit{Threshold} (Filtro Threshold): 
\\\textit{Hue Division} (Filtro Hue Division):
\\\textit{Threshold H} (Filtro Multiple Threshold):
\\\textit{Threshold S} (Filtro Multiple Threshold):
\\\textit{Threshold V} (Filtro Multiple Threshold):
\newline
\\\underline{Composition Config:}
\\En esta pestaña se pueden encontrar todas las opciones disponibles para el compositor musical, estás serán transmitidas a los modulos de la aplicacióncuando se pulse el boton de "Compose" en la "Main Window" a través del documento XML mencionado anteriormente.
\\Como se puede ver en la imagen adjunta, hay opciones para cuatro voces, que son aquellas con las que trabajan los compositores de la aplicación siendo, normalmente, la "Voice 1" la melodía principal, la "Voice 2" un acompañamiento, la "Voice 3" el bajo y la "Voice 4" la percusión. Las opciones disponibles son las siguientes:
\\\todo{Adjuntar imagen}
\newline
\\textit{Color System:} Hace referencia referencia a la teoría sinestésica que se utilizará como base para la relación de color-notas durante la composición musical.
\\textit{Composer:} En las cuatro voces se refiere a que compositor se utilizará para esa voz, al presionarlo se desplegarán los distintos compositores que estan disponibles para que el usuario elija el que mejor le convenga.
\\textit{Instrument:} En las cuatro voces se refiere a que instrumento se utilizará para esa voz en concreto, al presionarlo se desplegarán los instrumentos disponibles.
\\textit {Composer Mixer:} Hace referencia a que como se combinarán las cuatro voces anteriores.
\\textit{Tempo:} 
\\\todo{A rellenar con gente que sepa describirlo mejor que yo. \\También hay que repasar lo anterior}
\newline
\\{\bf Libreria Phonon}
\\Esta es una librería proporcionada por Qt para la reproducción de audio, al ser un módulo externo requiere una serie de librerias añadidas que están incluidas en el paquete de instalación de windows, sin embargo tendrán que ser instaladas en linux utilizando alguno de los gestores de software disponibles realizando una busqueda con la palabra clave: Phonon.
\\Las librerías requeridas son las siguientes: (LISTA DE LIBRERIAS)
\\En el caso de que no funcione el sistema de reproducción, se pueden encontrar los archivos de audio en la ruta especificada en "Midi Output"
\newline
\\\todo{Claramente esta parte va en arquitectura, sin embargo no se que más comentar de la arquitectura de la GUI puesto que casi todo esta hecho con QT, como no digamos el widget de carlos...}
\\La interacción con phonon se realiza utilizando la funcionalidad proporcionada por el framework. Se asocia el fichero multimedia a un tipo proporcionado por Phonon que a su vez lo conecta con el sistema de audio predeterminado para cada sistema operativo.
\\El uso de Phonon es conveniente porque a pesar de sus complicaciones ofrece una manera sencilla de incluir un reproductor en la interfaz, que a su vez, es multiplataforma sin obligar al usuario a buscar el archivo de audio o forzar una llamada a un tercer programa que reproducir la musica que generada.