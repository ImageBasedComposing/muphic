\section{Representación de la imagen analizada}

Tanto en el análisis de imágenes como en el proceso de composición hay un aspecto fundamental a estudiar: la representación de datos de las figuras.\\

De acuerdo con lo explicado en previas secciones, el producto de salida  (y el de entrada de la composición musical) es un archivo XML que almacena el análisis imagen dada como entrada. Este archivo contiene toda la información que se considera para representar los elementos de una imagen, y servirá como base para explicar el contenido y estructura lógica de los datos analizados.\\

Por otro lado, en esta sección se mostrará también la estructura usada en la implementación de los objetos relacionados con estos datos y sus usos. Dado que tanto el módulo de análisis como el de composición hacen uso de esta información, se introducirá también las extensiones y posibilidades de la estructura de datos creada.

\subsection{Almacenamiento y estructura}
\label{subsec:xmlstruct}

	La información correspondiente a la salida del análisis de imágenes es tal y como muestra la Figura~\ref{fig:estructuraFiguras}.\\

			\begin{figure}[htbp]
			\centering
			\includegraphics[scale=0.47]{graphics/todo.png}
			\caption{Estructura del archivo xml de análisis}
			\label{fig:estructuraFiguras}
			\end{figure}
		
	La estructura (representada en formato XML) es:\\
	
	\newlist{longenum}{enumerate}{5}
	\setlist[longenum,1]{label=\tiny$\blacksquare$}
	\setlist[longenum,2]{label=\tiny$\blacksquare$}
	\setlist[longenum,3]{label=\tiny$\blacksquare$}
	\setlist[longenum,4]{label=\tiny$\blacksquare$}
	\setlist[longenum,5]{label=\tiny$\blacksquare$}
	
	\begin{longenum}
	\item \textbf{Shapes}: Representa la imagen completa y contiene un listado de las figuras resultantes del análisis.
	\item \textbf{Width}: Almacena el ancho total de la imagen.
	\item \textbf{Height}: Almacena el alto total de la imagen.
	\item \textbf{Figure}: Contiene la información relativa a un polígono en la imagen.
		\begin{longenum}
		\item \textbf{Id}: Identificador único de la figura.
		\item \textbf{Color}: Identificador para almacenar la información sobre el color de la figura.
			\begin{longenum}
			\item \textbf {RGB}: Delimitar la información sobre los valores RGB del color.
				\begin{longenum}
				\item \textbf{R}: Valor referente a la cantidad de rojo en la escala RGB.
				\item \textbf{G}: Valor referente a la cantidad de verde en la escala RGB.
				\item \textbf{B}: Valor referente a la cantidad de azul presente en escala RGB.
				\end{longenum}
			\end{longenum}
		\item \textbf{VertexList}: Lista de vértices que componen el polígono.
			\begin{longenum}
			\item \textbf{Vertex [type=``normal'']}: Identificador para delimitar un vértice que pertenece a uno de los extremos de una recta o curva del polígono.
				\begin{longenum}
					\item \textbf{Position}: Identificador para delimitar la posición de un vértice en la imagen.
						\begin{longenum}
						\item \textbf{X}: Componente x de la posición del vértice.
						\item \textbf{Y}: Componente y de la posición del vértice.
						\end{longenum}
					\item \textbf{Vertex [type=``center'']}: Identificador para delimitar un vértice que señala el centro de circunferencia que forma la curva que conecta el vértice anterior a este con el siguiente.
				\end{longenum}
			\end{longenum}
		\end{longenum}
		\item \textbf{Area}: Valor numérico del área de una figura.
		\item \textbf{Canvas}: Identificador que delimita las figuras situadas dentro de la figura que contiene este elemento. 
	\end{longenum}
	
	Como se puede ver, la imagen analizada está compuesta por una lista de figuras (o polígonos) jerarquidas según su posición geométrica. Cada figura guarda información además de sus vértices, color y area, y representa una mancha de color detectada por el analizador de imágenes.


\subsection{Vista de Implementación}

	La vista principal de la estructura en un sistema basado en objetos como el tratado se muestra en la Figura~\ref{fig:diagramaClasesFigure}.\\

		\begin{figure}[htbp]
		\centering
		\includegraphics[scale=0.47]{graphics/todo.png}
		\caption{Diagrama de clases de la estructura de datos \emph{Figuras}}
		\label{fig:diagramaClasesFigure}
		\end{figure}
		
		\todo{este diagrama debe contener variables como color, x, y, area y demás, para reflejar la información mostrada arriba}
		
	Como se puede ver, se ha implementado mediante tres clases principales:
	
	\begin{itemize}
	
		\item \textbf{Figures:} Almacena la lista de figuras presentes en la imagen, y permite manipular y acceder a ellas tanto en forma de lista (colocándose todas las figuras existentes de forma seguida y sin jerarquías) como en forma de árbol (jerarquizándolas según quién está dentro de quién: una figura que contiene en su interior todos los vértices de otra será su ``padre''), ocupandose de forma autosuficiente de mantener ésta última estructura al añadir figuras nuevas.\\
	
		Posee funcionalidad para guardar y cargar su información en formato XML, así como para eliminar repetidos y recalcular el color de cada figura teniendo en cuenta las figuras que contiene (ver Sección~\ref{sec:algAnalisis} para más detalles sobre estas funcionalidades nombradas).
		
		\item \textbf{Figure:} Representa cada elemento almacenado en la clase Figures. Contiene, como nodo de la estructura arbórea creada en Figures, los atributos necesarios para referenciar a sus padres, sus hijos y sus hermanos, y como elemento representativo de una figura en la imagen (Sección~\ref{subsec:xmlstruct}), el color, area y lista de vértices correspondiente a la imagen.\\
		
		Presenta métodos para poder acarrear funciones geométricas básicas (como transformar su representación de vértices de coordenadas cartesianas a euclídeas) que simplifican el desarrollo de los algoritmos explicados en la Sección~\ref{sec:algComposicion}.
		
		\item \textbf{Vertice:} Se trata de la unidad más pequeña de esta estructura, y representa las coordenadas en 2 dimensiones de un vértice en el plano cartesiano. Como se detalla en la Sección~\ref{subsec:xmlstruct}, da la opción de ser considerado un ``centro'', de forma que, en la lista de vértices de la Figura que lo almacene, puede actuar como el centro entre el vértice anterior y el siguiente, formando por tanto un arco entre los 3 vértices y permitiendo realizar curvas en las figuras.
		
		\item \textbf{TinyXML:} Se trata de un paquete externo de código libre que permite generar y manipular archivos XML.
	
	\end{itemize}
	
			
\subsection{Usos de la estructura}	

	Como ya se ha comentado, tanto el proceso de análisis como el de composición hacen uso de esta estructura y, dado que para la implementación de ambos módulos se ha seleccionado el mismo lenguaje orientado a objetos (C++), estos compartirán la implementación de esta estructura.\\
	
	Sin embargo, existen diferencias entre los usos y ampliaciones que cada módulo aplica, y es por ello que no hacen uso de la estructura de forma directa,  sino que cada módulo heredará la clase \emph{Figure} para satisfacer sus propias necesidades. El resultado es el diagrama mostrado en la Figura~\ref{fig:diagramaFigMupPhic}, con diferenciación entre \emph{FigureImg} y \emph{FigureMusic}.\\

		\begin{figure}[htbp]
		\centering
		\includegraphics[scale=0.47]{graphics/todo.png}
		\caption{Diagrama de clases de los diferentes tipos de Figura}
		\label{fig:diagramaFigMupPhic}
		\end{figure}
		
	Los detalles de cada clase hija serán explicados en posteriores secciones.	


\section{Formato de representación de música}

\torev{Última revisión:; 05-06-1012}

La estructura para representar la música propuesta en el proyecto está determinada por un árbol que trabaja desde el elemento más general, la canción que se va a componer, al más específico, cada una de las notas que componen dicha canción, como muestra la Figura~\ref{fig:structmusic}.\\
	
	\begin{figure}[htbp]
	\centering
	\hspace*{-0.1in}
	\includegraphics[scale=0.47]{graphics/musica-estructura.png}
	\caption{\color{blue}Estructura de una pieza musical\color{black}}
	\label{fig:structmusic}
	\end{figure}


\textbf{Música}: Al nivel de la Música, trabajamos con la información relativa a toda la canción que se va a componer, por un lado está información relativa al nombre de la composición y del compositor que la ha hecho, también dispone de una serie de voces a partir de las cuales estará formada la canción y su tempo. Por último a un nivel más cercano, la implementación permitirá elegir con que herramienta querremos crear el archivo de audio de salida.
\newline

\textbf{Voz}: Por debajo de la Música se trabaja con las Voces. Su estructura permite determinar el instrumento, la tonalidad y\color{blue} el tempo de cada una de las voces que componen una pieza\color{black}, cada una de estas voces esta compuesta de varios segmentos musicales, los cuales al ser reproducidos en el orden establecido por el compositor forman la Voz en cuestión.
\newline

\textbf{Segmento}: Estos elementos tienen la posibilidad de establecer su propia métrica, su tempo y su duración. Los segmentos a su vez están formados por una sucesión de símbolos.
\newline

\textbf{Símbolo}: Son la unidad a partir\color{blue} de la cual \color{black}se crean todos los elementos básicos de la música, símbolo nos permite establecer una duración para estos elementos. Actualmente el proyecto crea dos tipos de elementos a partir de símbolo:
\begin{itemize}
\item \textbf{Notas}: Estos símbolos tienen asociada una duración, determinada por el componente anterior, y un tono establecido por un número que posteriormente será interpretado con la herramienta que generará el archivo de audio.
\item \textbf{Acordes}: Los acordes están formados de la misma manera que las notas pero en lugar de tener asociado un único tono pueden tener de dos a tres tonos diferentes, en la práctica esto resultará en todos los tonos sonando a la vez durante el mismo tiempo una vez generada la canción.
\end{itemize}
 
