\chapter{Sistema}

\section{Requisitos}
\todo{Ampliar, desarrollar, repasar y juntar las que sean redundantes o juntables (valgame la redundancia)}
Los usuarios con los que trabajaremos para delimitar los requisitos son:
\begin{itemize}
	\item Usuario: Personas con un conocimiento mínimo o nulo sobré música y ofimática que hayan leído o se les haya explicado el funcionamiento de la aplicación
	\item Desarrollador: Personas con un conocimiento amplio de programación y con conocimiento básico o avanzado de música.
\end{itemize}
Los requisitos funcionales del proyecto son:
\\Para la interfaz gráfica:
 \begin{itemize}
	 \item La interfaz deberá ser capaz de interpretar archivos de imagen del formato .png y mostrarlos por pantalla
	 \item La interfaz deberá ser capaz de interpretar archivos de audio del formato .wav y permitir funciones básicas de reproducción sobre los mismos
	 \item La interfaz deberá ser capaz de generar un archivo de configuración interpretable por tanto Mu como Phic a partir de los parámetros de entrada establecidos por el usuario.
	 \item La interfaz deberá de ser capaz de, mediante una llamada a otro programa, mostrar la partitura de la música compuesta tras haber sido generada.
	 \item La interfaz deberá controlar posibles errores cometidos por el usuario por tocar parámetros no disponibles para la configuración seleccionada, a su vez deberá de limitar las acciones del usuario a la hora de modificar dichos parámetros según las configuraciones elegidas.
 \end{itemize}
 Para Phic:
 \begin{itemize}
	\item El analizador deberá recibir un archivo de configuración que determine como realizar el análisis y un archivo de imagen que analizar
	\item El analizador deberá generar un archivo XML con los siguientes datos:
	\begin{itemize}
		\item Los vértices de las figuras que componen la imagen de entrada y el número total de vértices por figura.
		\item Los  colores que contienen dichas figuras en formato RGB
		\item Las figuras estarán organizadas jerárquicamente al igual que se presentan en la imagen, es decir, si una figura está dentro de otra en el XML se verá reflejado siendo la segunda figura un elemento incluido en la primera.
	\end{itemize}
	\item El analizador deberá de devolver una imagen con la representación poligonal de la imagen de entrada.
 \end{itemize}
 Para Mu:
 \begin{itemize}
	\item El compositor deberá de recibir un archivo de configuración y un archivo con los resultados del análisis correctamente estructurados.
	\item El compositor deberá generar un archivo abc interpretable por cualquier programa que pueda recibir como entrada dicho tipo de archivos.
	\item El compositor deberá de ser capaz de llamar a terceros que transformen el archivo abc generado a los formatos midi y wav.
 \end{itemize}
Requisitos no funcionales:
\\Generales o/y GUI:
\begin{itemize}
	\item La aplicación facilitará al usuario la entrada de datos de configuración o imágenes que desee analizar.
	\item La aplicación facilitará al usuario los resultados del análisis hecho de la imagén de entrada.
	\item La aplicación producirá archivos de audio como resultados de la composición capaces de ser reproducidos en cualquier sistema con soporte básico de audio.
	\item La aplicación facilitará la reproducción de estos archivos al usuario con un reproductor de música incrustado.
	\item La aplicación facilitará el establecimiento del directorio destino de los archivos de audio.
	\item La aplicación facilitará la partitura compuesta por ella misma tras haberse generado la música.
\end{itemize}
Phic:
\begin{itemize}
	\item Al menos uno de los analizadores devolverá un análisis que se asemeje a ''primera vista'' a la imagen original
\end{itemize}
Mu:
\begin{itemize}
	\item El compositor mediante su estructura interna facilitará al desarrollador la creación de nuevos algoritmos de composición.
	\item El compositor principal de la aplicación deberá de ser capaz de generar una música que no tenga porque ser ''interesante'' pero que ''no sonará mal'' en ambos casos los resultados en ultima instancia dependerán de la opinión subjetiva del usuario
\end{itemize}
\todo{Pendiente de hacerse, ideas:\\

	requisitos\\
	\begin{itemize}
		\item analizar imágenes a un formato directo de composición
		componer análisis
		\item permitir variar parámetros de análisis
		\item permitir variar parámetros de composición
		\item multiplataforma\\
	\end{itemize}
		
		
	restricciones\\
	\begin{itemize}
		\item el sistema debe ser lo suficientemente flexible como para permitir expandirse con otros lenguajes
		\item el análisis y composición no deben consumir muchos recursos\\
	\end{itemize}
		
	requisitos no funcionales (nombrados en la introducción)\\
	\begin{itemize}	
		\item No se tienen en cuenta objetos físicos en el análisis, solo color
		\item El único componente de entrada usado es la imagen dada, nada de música externa ya creada.
		\item Generación de contenido frente a acoplamiento de creaciones preestablecidas	\\	
	\end{itemize}		
		
		
Casos de uso?

	(análisis: explicar proceso de filtrado y en sistema explicar solo la arquitectura de clases?)
}
\section{Arquitectura}

\subsection{Formato de representación de música}

\todo{Repasar formalizar}

La estructura para representar la música propuesta en el proyecto está determinada por un árbol que trabaja desde el elemento más general, la canción que se va a componer, al más específico, cada una de las notas que componen dicha canción, como muestra la Figura~\ref{fig:structmusic}.\\
	
	\begin{figure}[htbp]
	\centering
	\hspace*{-0.1in}
	\includegraphics[scale=0.47]{graphics/musica-estructura.png}
	\caption{Estructura de una pieza musical}
	\label{fig:structmusic}
	\end{figure}


\textbf{Música}: Al nivel de la Música, trabajamos con la información relativa a toda la canción que se va a componer, por un lado está información relativa al nombre de la composición y del compositor que la ha hecho, también dispone de una serie de voces a partir de las cuales estará formada la canción y su tempo. Por último a un nivel más cercano, la implementación permitirá elegir con que herramienta querremos crear el archivo de audio de salida.
\newline

\textbf{Voz}: Por debajo de la Música se trabaja con las Voces. Su estructura permite determinar el instrumento, la tonalidad y el tempo de cada una de las voces que componen una pieza, cada una de estas voces esta compuesta de varios segmentos musicales, los cuales al ser reproducidos en el orden establecido por el compositor forman la Voz en cuestión.
\newline

\textbf{Segmento}: Estos elementos tienen la posibilidad de establecer su propia métrica, su tempo y su duración. Los segmentos a su vez están formados por una sucesión de símbolos.
\newline

\textbf{Símbolo}: Son la unidad a partir de la cual se crean todos los elementos básicos de la música, símbolo nos permite establecer una duración para estos elementos. Actualmente el proyecto crea dos tipos de elementos a partir de símbolo:
\begin{itemize}
\item \textbf{Notas}: Estos símbolos tienen asociada una duración, determinada por el componente anterior, y un tono establecido por un número que posteriormente será interpretado con la herramienta que generará el archivo de audio.
\item \textbf{Acordes}: Los acordes están formados de la misma manera que las notas pero en lugar de tener asociado un único tono pueden tener de dos a tres tonos diferentes, en la práctica esto resultará en todos los tonos sonando a la vez durante el mismo tiempo una vez generada la canción.
\end{itemize}

\subsection{Formato de representación de imágenes}

\todo{está en google docs, sólo es pasarlo a limpio}

\subsection{Sistema de análisis de Imágenes: Phic}

\todo{hacer}

\subsection{Sistema de composición algorítmica: Mu}

\todo{hacer}

\subsection{Sistema de enlace de módulos: Muphic}

\todo{hacer}

\subsubsection{Interfaz Gráfica}

El desarrollo de la interfáz gráfica se decidió realizar a través de un framework que facilitase la construcción de la misma. Qt, la herramienta que se utilizó se eligió debido a dos criterios, era multiplataforma, al igual que la aplicación, servía para trabajar con móviles, siguiendo así uno de los enfoques iniciales del proyecto que más adelante fue descartado. Además Qt trabaja con C++ como lenguaje de programación, el mismo que el núcleo de la aplicación y ofrece facilidades a la hora de integrar audio, a través de la librería Phonon, o componentes personalizados en una interfaz.\\
\todo{Este párrafo no se si abriría mejor la parte de arquitectura}
\newline
La interfáz esta compuesta de tres pestañas cada una encargada de una parte de la funcionalidad:
\newline
\\\underline{Graphic Config:}
\\En esta pestaña se encuentran todas las opciones relacionadas con el análisis de la imagen, las distintas configuraciones se transmitirán a los módulos de la aplicación a través de un documento XML que se genera cuando en la "Main Window" se le da al boton de "Analyze".
\\Los distintos componentes de esta pestaña van variando según el filtro que se seleccione, sin embargo algunos de ellos son comunes para todos los filtros:
\newline
\\\textit{Filter:} Este combo box permite seleccionar entre los distintos filtros que se pueden usar en la aplicación.
\\\textit{Noise Selection:} Con esta barra de desplazamiento se podrá elegir a partir de que tamaño, relativo al area del mayor polígono de la imagen, se ignorarán los poligonos encontrados durante el análisis de la imagen.
\\\textit{Analysis Depth:} Esta barra permite seleccionar cuanto deseamos comprimir la imagen original antes de analizarla, cuando menor sea su valor menor será el nivel de detalle obtenido y más rápido se realizará el análisis.
\\\textit{Polygon Simplification:} Esta barra permite seleccionar cuanto deseamos simplificar los poligonos obtenidos a partir de la imagen original, cuanto mayor sea su valor menos vertices tendrán los poligonos obtenidos y por lo tanto menos fiel será el resultado del análisis, sin embargo, este se realizará más rápido.
\newline
\\Según el filtro que se elija aparecerán mas componentes:
\newline
\\\todo{Rellenar esto de manera experta y didáctica}
\\\textit{Threshold} (Filtro Threshold): 
\\\textit{Hue Division} (Filtro Hue Division):
\\\textit{Threshold H} (Filtro Multiple Threshold):
\\\textit{Threshold S} (Filtro Multiple Threshold):
\\\textit{Threshold V} (Filtro Multiple Threshold):
\newline
\\\underline{Composition Config:}
\\En esta pestaña se pueden encontrar todas las opciones disponibles para el compositor musical, estás serán transmitidas a los modulos de la aplicacióncuando se pulse el boton de "Compose" en la "Main Window" a través del documento XML mencionado anteriormente.
\\Como se puede ver en la imagen adjunta, hay opciones para cuatro voces, que son aquellas con las que trabajan los compositores de la aplicación siendo, normalmente, la "Voice 1" la melodía principal, la "Voice 2" un acompañamiento, la "Voice 3" el bajo y la "Voice 4" la percusión. Las opciones disponibles son las siguientes:
\\\todo{Adjuntar imagen}
\newline
\\textit{Color System:} Hace referencia referencia a la teoría sinestésica que se utilizará como base para la relación de color-notas durante la composición musical.
\\textit{Composer:} En las cuatro voces se refiere a que compositor se utilizará para esa voz, al presionarlo se desplegarán los distintos compositores que estan disponibles para que el usuario elija el que mejor le convenga.
\\textit{Instrument:} En las cuatro voces se refiere a que instrumento se utilizará para esa voz en concreto, al presionarlo se desplegarán los instrumentos disponibles.
\\textit {Composer Mixer:} Hace referencia a que como se combinarán las cuatro voces anteriores.
\\textit{Tempo:} 
\\\todo{A rellenar con gente que sepa describirlo mejor que yo. \\También hay que repasar lo anterior}
\newline
\\{\bf Libreria Phonon}
\\Esta es una librería proporcionada por Qt para la reproducción de audio, al ser un módulo externo requiere una serie de librerias añadidas que están incluidas en el paquete de instalación de windows, sin embargo tendrán que ser instaladas en linux utilizando alguno de los gestores de software disponibles realizando una busqueda con la palabra clave: Phonon.
\\Las librerías requeridas son las siguientes: (LISTA DE LIBRERIAS)
\\En el caso de que no funcione el sistema de reproducción, se pueden encontrar los archivos de audio en la ruta especificada en "Midi Output"
\newline
\\\todo{Claramente esta parte va en arquitectura, sin embargo no se que más comentar de la arquitectura de la GUI puesto que casi todo esta hecho con QT, como no digamos el widget de carlos...}
\\La interacción con phonon se realiza utilizando la funcionalidad proporcionada por el framework. Se asocia el fichero multimedia a un tipo proporcionado por Phonon que a su vez lo conecta con el sistema de audio predeterminado para cada sistema operativo.
\\El uso de Phonon es conveniente porque a pesar de sus complicaciones ofrece una manera sencilla de incluir un reproductor en la interfaz, que a su vez, es multiplataforma sin obligar al usuario a buscar el archivo de audio o forzar una llamada a un tercer programa que reproducir la musica que generada.
\section{Futuras ampliaciones}

\todo{Esto es solo un boceto inicial, repasar, ampliar, corregir, dar formato etc...}
\todo{Ideas planteadas:
\\Portar a móviles
\\Crear compositores nuevos
}
Inicialmente el proyecto fue planteado como una aplicación para dispositivos móviles, tanto debido a la falta de tiempo como a la complicación que suponía adaptar las librerías utilizadas a distintos sistemas operativos esta parte quedó aplazada como una posible ampliación.
\newline
El trabajo a realizar consistiría en portar OpenCV, la librería encargada del tratamiento de imágenes, a android y/o iPhone, buscar una librería o un app que trabajase con abc y midi en dichos dispositivos, esta parte no debería ser complicada pues, por ejemplo, existen muchos recursos puestos a disposición por la comunidad de java, lenguaje que gracias a que trabaja a través de una máquina virtual es compatible con los sistemas operativos de ambos tipos de dispositivos.
\\Y por último modificar la presentación de la interfaz gráfica para que sea mas manejable con dispositivos móviles. Dado que Qt es un framework diseñado para trabajar con este tipo de dispositivos la portabilidad no sería un problema.
\newline
Otra ampliación disponible y para la que fue diseñado originalmente el proyecto es la posibilidad de hacer más compositores.
\\Según el diseño planteado habrá que tener en cuenta tres elementos:
\begin{itemize}

\item El o los compositores que se desean hacer:
\\El desarrollador tendrá la libertad de elegir si simplemente desea añadir un compositor a los mezcladores ya creados o construir un ''compositor completo'', es decir, melodía con varias voces, bajo y ritmo. (p.e. Uno de melodía y uno de ritmo, dos de ritmo uno de melodía...)

\item El o los compositores que se vayan a encargar de juntar los compositores creados:
\\Con esto nos referimos a la clase que se encarga de juntar las melodías creadas por los compositores anteriores para que suene correctamente, o la manera en la que se interpretarán los compositores creados a la hora de generar el archivo abc. También podría tenerse en cuenta a la hora de generar el archivo de audio si no se quiere que este sea un midi.

\item La inclusión en la interfaz gráfica de quererse usar:
\\Al estar diseñado de manera modular si el usuario prefiere utilizar la aplicación a través de una consola en lugar 	de la interfaz que hemos creado es libre de hacerlo por lo que no tendría porque añadir su compositor a la interfaz gráfica ofrecida.

\end{itemize} 

Para crear tanto los compositores como el mezclador deberá de crearse clases que hereden de Composer y a partir de ahí programar la lógica que se haya planteado previamente.
Para la interfaz necesitará tener instalado en su sistema el Qt SDK y simplemente añadir una entrada más a la lista de compositores, y a la de mezcladores. Podrá asociar su compositor a un único mezclador en la interfaz, con lo que deberá en la clase principal de Mu añadir la referencia al mezclador en cuestión con el índice que le haya asignado, o dejarlo asociado a todos los mezcladores, con lo que tendrá que añadir las referencias a su/s compositor/es a todos los mezcladores que hay en Mu.