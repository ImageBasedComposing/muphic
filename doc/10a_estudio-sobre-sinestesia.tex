\subsection{Estudio sobre Sinestesia}
\label{subsubsec:estudioSinestesia}

\torev{Última revisión realizada: 5-06-2012}\\

Dentro de la sinestesia, la parte que es importante en este proyecto es la llamada sinestesia musical. Esta consiste en mezclar la experiencia auditiva con la visual. \color{blue}Se ha estudiado este tipo de sinestesia porque parece lógico pensar que será una fuente de información razonable para poder empezar a realizar la conversión de imagen a música. \color{black}
Según el neurocientífico David Eagleman, las personas de forma innata tienen la tendencia de conectar sentidos, entre otros visual y auditivo (sonidos con formas y sonidos con colores, \cite{VideoRedesFliparColores}).\\

Varios artistas y científicos en el tiempo han expresado su capacidad sinestésica a través de sus obras o documentando y experimentando con este fenómeno. Desde la antigua Grecia se intentaba encontrar esa equivalencia entre los colores y los sonidos. Aristóteles en su ensayo (\emph{De Sensu et Sensato} \cite{DeSensuEtSensato}, 439b30) realiza una descripción de los colores comparándolos directamente con la armonía presente en la música. \color{blue}Otra teoría dentro de la vertiente científica-filosófica, Newton asignó un color a cada nota de la escala diatónica (siete notas) siguiendo los colores que dividía su prisma que también eran siete tal que empezando por C (Do) es rojo hasta B (Si) que le corresponde violeta (\cite{OpticksNewton}).\\ \color{black}

Otra aproximación a la relación música-imagen que normalmente siguen los artistas es a través de la experiencia con el fenómeno de la sinestesia. Encontramos entre ellos al compositor Scriabin, quien hizo una asociación entre los colores y los acordes que aparecen en música usando el círculo de quintas (\cite{ScriabinQuintasColor}). Kandinsky, otro artista pero pintor artístico, partiendo de la pintura ha investigado y defendido fervientemente la posibilidad de asociar la música a la pintura y su relación directa (\cite{ConcerningSpiritualArt}). \color{blue} Resalta aspectos como que la escala de intensidad de los sonidos está relacionada con la intensidad de los trazos de la pintura, también distingue timbres de instrumentos por colores asignando a cada familia colores parecidos.\\ \color{black}

Las investigaciones llevadas a cabo por el psicólogo Köhler en 1929 (\cite{GestaltPsychology}) y las hechas recientemente demuestran que tenemos una conexión profunda entre los sonidos y las formas, además de los colores. Marks, en \emph{The Unity of the Senses} (\cite{TheUnityOfTheSenses}) encuentra una asociación entre los tiempos en música y las formas, tal que cuanto más angular e irregular es una figura \color{blue} más rápidas deben ser las notas o el tempo. \color{black} También declara sobre los colores que:
\begin{quote}
\emph{``Por desgracia, al final uno descubre que no hay una asociación sinestésica entre notas musicales y colores que prevalezca ante las demás''}.\\
\end{quote}

\color{blue} 
Después de ver la aproximación a través de la experiencia (con los trabajos de Scriabin o Kardinsky ya mencionados) a través de la lógica o fundamentación matemática (como los propuestos por Aristótles, Newton o Köhler), queda claro que cada persona sinestésica es única y puede haber grandes diferencias entre unas experiencias sinestésicas y otras. Es por tanto importante poder encontrar asociaciones aceptables y ponerlas en práctica para comprobar los distintos resultados. Para poder probar las diferentes correspondencias entre música e imagen se ha habilitado en la herramienta el poder elegir diferentes asociaciones entras las citadas en esta sección y más autores. \color{black}