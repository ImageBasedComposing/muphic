\subsubsection{Estudio sobre Sinestesia}

Dentro de la sinestesia, la parte que es importante en nuestro proyecto es la llamada sinestesia musical. Aunque no es nuestro objetivo realizar una investigación rigurosa y llegar a una conclusión o afirmar una suposición realizada con antelación, estudiamos la sinestesia porque pensamos que sería una aproximación muy buena para poder empezar a realizar la conversión de imagen a música.
Según (Video Redes:Flipar en Colores)David Eagleman las personas de forma innata  tenemos la tendencia de conectar sentidos, entre otros visual y auditivo (sonidos con formas y sonidos con colores)
Varios artistas en el tiempo han expresado su capacidad sinestésica a través de sus obras o documentando y experimentando con este fenómeno. Entre ellos se encuentra Scriabin, quien hizo una asociación entre los colores y los acordes que aparecen en música de forma rigurosa. También hay científicos como Newton que intentaron asignar colores a la notas pero no es un problema reciente, ya desde la antigua Grecia con Aristóteles y Pitágoras se lleva intentando dar una solución definitiva. Después de ver la aproximación a través de la experiencia o a través de la lógica o fundamentación matemática queda claro que cada persona sinestésica es única puede haber grandes diferencias entre unas y otras.
Las investigaciones llevadas a cabo por el psicólogo Wolfgang Köhler en 1929 y las hechas recientemente demuestran que tenemos una conexión profunda entre los sonidos y las formas de lo que vemos. (// creo mejor no hablar de esto:Su experimento se basaba en dejar asignar dos palabras (sonidos) a dos figuras//). Es por tanto importante poder encontrar esa asociación y ponerla en práctica a la hora de convertir las imágenes en música. Lawrence E. Marks, The Unity of the Senses encuentra una asociación entre los tiempos en música y las formas, que cuanto más angular e irregular es una figura el tempo o las notas son más rápidas.