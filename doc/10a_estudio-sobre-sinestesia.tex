\subsubsection{Estudio sobre Sinestesia}
\label{subsubsec:estudioSinestesia}

Dentro de la sinestesia, la parte que es importante en este proyecto es la llamada sinestesia musical. Aunque no es el objetivo realizar una investigación rigurosa y llegar a una conclusión o afirmar una suposición realizada con antelación, se ha estudiado la sinestesia porque cabe pensar que sería una aproximación muy buena para poder empezar a realizar la conversión de imagen a música.
Según \emph{(Video Redes:Flipar en Colores)}David Eagleman las personas de forma innata tienen la tendencia de conectar sentidos, entre otros visual y auditivo (sonidos con formas y sonidos con colores).\\

Varios artistas en el tiempo han expresado su capacidad sinestésica a través de sus obras o documentando y experimentando con este fenómeno. Entre ellos se encuentra Scriabin, quien hizo una asociación entre los colores y los acordes que aparecen en música de forma rigurosa. También hay científicos como Newton que intentaron asignar colores a la notas, pero no es un problema reciente, ya desde la antigua Grecia con Aristóteles y Pitágoras se lleva intentando dar una solución definitiva. Después de ver la aproximación a través de la experiencia o a través de la lógica o fundamentación matemática queda claro que cada persona sinestésica es única y puede haber grandes diferencias entre unas experiencias sinestésicas y otras.\\

Las investigaciones llevadas a cabo por el psicólogo Wolfgang Köhler en 1929 y las hechas recientemente demuestran que tenemos una conexión profunda entre los sonidos y las formas de lo que vemos. 

\todo{creo mejor no hablar de esto:Su experimento se basaba en dejar asignar dos palabras (sonidos) a dos figuras...kiki-booma}

 Es por tanto importante poder encontrar esa asociación y ponerla en práctica a la hora de convertir las imágenes en música. Lawrence E. Marks, The Unity of the Senses encuentra una asociación entre los tiempos en música y las formas, que cuanto más angular e irregular es una figura el tempo o las notas son más rápidas.También declara sobre los colores que \emph{``Sadly, one finds that synesthetic associations between colors and musical notes fail to favor any particular scheme over the others.''}: Por desgracia, al final uno descubre que no hay una asociación sinestésica entre notas musicales y colores que prevalezca ante las demás.\\

Kandinsky, otro artista como Scriabin pero partiendo de la pintura, ha investigado y defendido fervientemente la posibilidad de asociar la música a la pintura.De varios de sus estudios y siguiendo sus propias experiencias sinestésicas ha escrito abundante documentación. Relaciona otros aspectos como por ejemplo que la escala de intensidad de los sonidos estaba relacionada con la intensidad de los trazos de la pintura, también distingue timbres de instrumentos por colores asignando a cada familia colores parecidos. \\
