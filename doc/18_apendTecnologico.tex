\chapter{Tecnologías utilizadas}

\section{Notación ABC}
\label{sec:NotacionABC}
\torev{Pendiente de primera revisión}

En 1993, Chris Walshaw introdujo una nueva notación musical de texto plano, junto con abc2mtex (programa que traduce la notación ABC en notación que usa MusicTeX y TeX). Como resultado se obtuvo un lenguaje para escribir partituras de música. Poco después, Michael Methfessel con abc2ps y James Allwright followed con abcMIDI, empezaron a exportar la notación ABC en diferentes formatos.

Inicialmente ABC se creó para escribir música tradicional. Gradualmente se fue añadiendo nuevas funcionalidades expandiendo sus posibilidades, pero ante todo se ha seguido preservando la mayor ventaja de esta notación: es fácil de crear, es ligera en almacenamiento (archivos de tamaño muy pequeño) y se puede compartir y transformar fácilmente, ya que es texto plano. Además, si está bien formateada la información se puede leer tras adquirir cierta experiencia.

Hoy en día hay un gran número de aplicaciones software que son compatibles con la notación ABC. Algunos programas comerciales permiten cargar o guardar la música producida en formato ABC. También existe la posibilidad de transformar del formato ABC a formato MIDI y viceversa. La oferta de software gratuito que maneja esta notación es amplia y variada. 

Gracias a todas estas características, la notación ABC se ha adoptado como solución al problema que surge de traducir la representación interna de la música dentro del sistema a formato de audio MIDI. Para conseguirlo se transforma de la notación musical interna a notación ABC y de ésta a un archivo MIDI gracias a abc2midi, que forma parte del proyecto abcMIDI.

\section{abcMIDI}
\label{sec:abcMIDI}
\torev{Pendiente de primera revisión}

abcMIDI consiste un paquete de programas desarrollado por James Allwright. Su cometido principal es procesar archivos con formato notación. Este paquete contiene los siguientes programas: abc2midi, abc2abc, yaps, abcm2ps y midi2abc. En el proyecto se usan tanto abc2midi (para producir archivos de audio), como abcm2ps (para producir partituras.

abc2midi sirve para convertir archivos con formato notación ABC en archivos de sonido MIDI. Es probablemente el programa más avanzado y maduro de las diferentes posibilidades de software libre disponible en la red. Además contiene funcionalidades extra como el manejo de archivos con múltiples voces, trasponer voces individuales y añadir un acompañamiento de percusión entre otros.

abcm2ps tiene como funcionalidad producir una partitura musical partiendo del archivo en notación ABC de entrada. Esta partitura puede darse en numerosos formados, siendo dos de ellos ps y xhtml. A diferencia de su versión previa, abcm2ps, permite trabajar con varias voces; aunque a día de hoy existen unas pocas características de la notación ABC que no es capaz de transformar a la correspondiente partitura.

\section{Timidity}
\label{sec:Timidity}

\section{TinyXML}
\label{sec:TinyXML}

\section{Qt}
\label{sec:Qt}

\section{Phonon}
\label{sec:Phonon}