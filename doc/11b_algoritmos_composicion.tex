\section{Algoritmos de Composición}
\label{sec:algComposicion}
\todo{
esquema\\
\begin{itemize}
\renewcommand{\labelitemi}{\tiny$\blacksquare$}
\item intro a la composición: separado por voces.
\item cada voz: con cada compositor que lo haga
\item coletilla de final sin decidir todavía
\end{itemize}
}

Para componer la música separamos en diferentes voces. De tal modo que la unión de todas las voces da como resultado una pieza musical. En el estado actual del proyecto hemos identificado 4 voces diferentes dando un papel especial a cada una. Cada una de ellas son un proceso independiente aunque deben concordar en algunos aspectos para la correcta unión de todas ellos como por ejemplo la duración total. Todos los compositores de forma imprescindible reciben una figura, en la que se van a basar para componer, y una duración que será el tiempo que dura el trozo de música.

%----------------------------------------------------------------------------------------------------------------------------------------------------
\subsection{1ª Voz: Melodía Principal}

Normalmente en una composición suele haber una voz que destaca de la demás, que es la que el oído reconoce primero. Esta voz reproduce la melodía. Para poder crear la melodía, hemos separado el proceso en cuatro fases: procesar la figura, obtener duraciones, obtener tonos y sintetizar. 
La primera fase que consiste en procesar la figura de entrada. Sacamos sus vértices ordenados y seguidamente los lados con sus longitudes de forma que la figura se puede leer desde arriba en sentido horario (Figura~\ref{fig:Figura1Voz1}).

		\begin{figure}[htbp]
		\centering
		\hspace*{0.0in}
		\includegraphics[scale=1.0]{graphics/simpletest1.png}
		\caption{Figura de entrada}
		\label{fig:Figura0Voz1}
		\end{figure}

		\begin{figure}[htbp]
		\centering
		\hspace*{0.0in}
		\includegraphics[scale=1.0]{graphics/simpletest1-F1.png}
		\caption{Figura de entrada con los vértices ordenados y con las longitudes de los lados}
		\label{fig:Figura1Voz1}
		\end{figure}

A continuación, en la segunda fase se calcula la longitud media de los lados y el lado más largo y más corto. Se asignan duraciones a cada lado teniendo en cuenta la duración máxima y la duración mínima con las longitudes recién calculadas (Figura~\ref{fig:Figura2Voz1}). Se usan las duraciones simples (figuras simples) de las notas en música obviando las compuestas (figuras compuestas), es decir, que se asignan duraciones de blancas, negras, corcheas, semicorcheas,... en vez de usar, por ejemplo, negra con doble puntillo que es una nota de duración negra más una corchea más una semicorchea. Buscar una equivalencia perfecta entre la longitud de los lados y la duración de las notas sería lo primero que pensaría el lector que podría ser la correspondencia más fiel, pero tras varias pruebas realizadas no se ha podido averiguar cómo hacerlo sin que se perdiera completamente el concepto de ritmo musical.

		\begin{figure}[htbp]
		\centering
		\hspace*{0.0in}
		\includegraphics[scale=1.0]{graphics/simpletest1-F2.png}
		\caption{Figura de entrada con el ritmo producido}
		\label{fig:Figura2Voz1}
		\end{figure}

La tercera fase se centra en sacar los tonos de las notas que vamos a crear. Se tienen varias formas diferentes de conseguirlo que además se han llevado a cabo en diferentes trabajos (\cite{bricksConvertsMusic} \cite{ImageBaseComposition}). La aproximación usada se basa en la idea de estabilidad en la que se centra el trabajo de A. Pintado (\cite{portutesis}). Una figura es estable cuanto más suave sea su contorno, es decir, cuantos menos picos o irregularidades tenga. A. Pintado usa esta cualidad para generar ritmos a partir de figuras y líneas, nosotros lo vamos a usar para calcular tonos.

La obtención de los tonos viene de los ángulos de las figura, para ello se coge el primer vértice y se calcula su ángulo. Por ser la primera nota le asignamos el tono correspondiente al color de la figura dentro del sistema de colores usado (p. ej: Rojo en el sistema Scriabin es C "do"). A partir de ahí dependiendo de cuanto ha variado el ángulo del siguiente vértice respecto al anterior vamos asignando un tono más alto o más bajo (si sube o baja la diferencia de ángulo) (Figura~\ref{fig:Figura4Voz1}).

De forma experimental y tras varias pruebas, hemos obtenido una correspondencia entre la variación de ángulos y la cantidad de tonos que debe variar el nuevo tono respecto al anterior (Figura~\ref{fig:Figura3Voz1}). De este modo si la diferencia de ángulos es menor a 10º se sigue usando el mismo tono, si la diferencia está entre 10º y 30º se usa el tono vecino dentro de la escala, si está entre 30º y 120º el segundo tono más cercano del acorde, entre 120º y 240º el segundo tono más cercano del acorde y hasta 360º el tercer tono más cercano del acorde. Usamos el acorde fundamental con tono fundamental la nota del color de la figura para los grandes saltos porque conseguimos notas consonantes con el color de la figura. Este hecho se basa en la asociación que tiene Scriabin con los colores(``Referencia Aquí''), él hacía corresponder un color a un tono y su acorde fundamental sin distinguir entre acorde mayor o menor para cada una de las notas de la escala cromática.

		\begin{figure}[htbp]
		\centering
		\hspace*{0.0in}
		\includegraphics[scale=0.75]{graphics/tabla-corresp-Tono-Angulo.png}
		\caption{Tabla correspondencias entre ángulos y cambios de tono}
		\label{fig:Figura3Voz1}
		\end{figure}

De esta manera conseguimos que una figura simétrica con todos sus ángulos iguales suene la misma nota ya que su estabilidad es muy alta. El círculo sería la figura de mayor estabilidad y se correspondería a una nota de larga duración, pero por el proceso de análisis, el círculo se aproxima a un polígono regular de muchos lados y por tanto a muchas notas con el mismo tono.

		\begin{figure}[htbp]
		\centering
		\hspace*{0.0in}
		\includegraphics[scale=1.0]{graphics/simpletest1-F3.png}
		\includegraphics[scale=1.0]{graphics/simpletest1-MELpartitura.png}
		\caption{Figura de entrada con los ángulos y los tonos que produce}
		\label{fig:Figura4Voz1}
		\end{figure}

Por último, la cuarta fase que combina las duraciones con los tonos para crear las notas. Al final hemos obtenido un trozo de música de una de las voces.

Este algoritmo está implementado en la clase ComposerFigMelody2. Además hemos dejado otro compositor ComposerFigMelody que cambia la segunda y tercera fase, fruto de las primeras pruebas en la composición. En la segunda fase, al hacer corresponder las duraciones con las longitudes de los lados, se usan las duraciones compuestas con unidad mínima indivisible la semicorchea o unidades más pequeñas si la duración pedida del segmento de música es especial (no divisible en semicorcheas) (Figura~\ref{fig:Figura5Voz1}).

		\begin{figure}[htbp]
		\centering
		\hspace*{0.0in}
		\includegraphics[scale=1]{graphics/simpletest1-F2_2.png}
		\caption{Figura de entrada con los lados y el ritmo conseguido}
		\label{fig:Figura5Voz1}
		\end{figure}

En la tercera fase, en vez de usar un algoritmo diferencial donde se tiene en cuenta la variación respecto al anterior ángulo examinado, se usa una correspondencia directa donde mayor águlo implica mayor salto de tono y el signo del ángulo es la dirección del salto (Figura~\ref{fig:Figura6Voz1}).

		\begin{figure}[htbp]
		\centering
		\hspace*{0.0in}
		\includegraphics[scale=1]{graphics/simpletest1-F3_2.png}
		\includegraphics[scale=1]{graphics/simpletest1_2-MELpartitura.png}
		\caption{Figura de entrada con los ángulos y los tonos producidos}
		\label{fig:Figura6Voz1}
		\end{figure}

%----------------------------------------------------------------------------------------------------------------------------------------------------

\subsection{2ª Voz: Melodía Secundaria}

Para crear el acompañamiento o segunda voz, se usa la misma estructura que a la hora de componer la melodía principal aunque también necesitamos el segmento de melodía principal como entrada para poder componer sobre ella. 
La primera fase es igual que en la primera voz. La segunda fase también es igual salvo que al final introducimos un paso de adaptación. Debemos hacer una adaptación de la duración del total obtenido al analizar la figura (Figura~\ref{fig:Figura1Voz2} sin adaptar y Figura~\ref{fig:Figura2Voz2} con adaptación). Eso ocurre ya que se quiere que el segmento de música que se genera tenga una duración determinada menor o igual a la melodía principal. Además se debe decidir en qué momento de la melodía principal se empieza a decorarla introduciendo la segunda voz.

 		\begin{figure}[htbp]
		\centering
		\hspace*{0.0in}
		\includegraphics[scale=1]{graphics/simpletest4-F2.png}
		\caption{Figuras de entrada con los lados de la figura interior y las duraciones obtenidas sin adaptación}
		\label{fig:Figura1Voz2}
		\end{figure}

		\begin{figure}[htbp]
		\centering
		\hspace*{0.0in}
		\includegraphics[scale=1]{graphics/simpletest4-F2_2.png}
		\caption{Figuras de entrada con la duración de la figura exterior y con las duraciones adaptadas de la interior}
		\label{fig:Figura2Voz2}
		\end{figure}

Esta función de adaptación se encarga de ir dividiendo a la mitad diferentes duraciones para reducir la duración total del segmento (o ir aumentando, duplicando duraciones, si se necesita aumentar la duración total). 

El otro cambio que hacemos es en la fase tercera. Mientras se generan los tonos se va analizando el comportamiento de la segunda voz respecto a la melodía principal. Este paso es necesario para disminuir las disonancias que puedan aparecer al juntar la primera y la segunda voz (Figura~\ref{fig:Figura3Voz2}). Por ello mientras se genera la nueva melodía se va haciendo pequeños cambios en la segunda voz que lleven a un resultado mejorado. Los cambios usados siguen los principios de las técnicas más básicas contrapuntísticas. Estos cambios se activan cuando el intervalo entre la voz primera y la voz segunda es disonante y se mantiene un periodo alargado en el tiempo (igual o mayor a la duración media de las notas, normalmente una negra). Cómo se intenta minimizar los cambios, se suele hacer que si el cambio de tono era para subir, se sube menos o más el nuevo tono dependiendo de qué implica un menor cambio. Lo mismo cuando el cambio de tono es para bajar, se baja un poco más o menos el nuevo tono (Figura~\ref{fig:Figura3Voz2}).

		\begin{figure}[htbp]
		\centering
		\hspace*{0.0in}
		\includegraphics[scale=1]{graphics/simpletest4-F3.png}
		\includegraphics[scale=1]{graphics/simpletest4-F3-MEL2partitura.png}
		\caption{Figuras de entrada con los ángulos y los tonos producidos sin técnicas contrapuntísticas}
		\label{fig:Figura3Voz2}
		\end{figure}

		\begin{figure}[htbp]
		\centering
		\hspace*{0.0in}
		\includegraphics[scale=1]{graphics/simpletest4-F3_2.png}
		\includegraphics[scale=1]{graphics/simpletest4-F3_2-MEL2partitura.png}
		\caption{Figuras de entrada con los ángulos y los tonos producidos con técnicas contrapuntísticas}
		\label{fig:Figura4Voz2}
		\end{figure}

En la mayoría de ocasiones, la segunda voz se pide una duración menor que el segmento de primera voz que va acompañandolo, esto ocurre porque normalmente se pasa una figura de entrada menor que la usada para componer la melodía principal. Para que el segmento tenga la misma duración que el segmento de la primera voz se rellenan los huecos (delante y/o detrás) del trozo de música compuesto con silencios.

%----------------------------------------------------------------------------------------------------------------------------------------------------

\subsection{3ª Voz: Bajo}

Para el bajo, se ha cogido un algortimo simple que consiste en notas de duración dada (por defecto redondas) que tengan como tono el color de la figura de entrada y que rellene la duración completa dada (Figura~\ref{fig:Figura1Voz3}).

		\begin{figure}[htbp]
		\centering
		\hspace*{0.0in}
		\includegraphics[scale=1]{graphics/simpletest2-F2F3.png}
		\includegraphics[scale=1]{graphics/simpletest2-BASSpartitura.png}
		\caption{Figura de entrada con los tonos producidos}
		\label{fig:Figura1Voz3}
		\end{figure}

Otra posibilidad habilitada para el bajo es usar la algoritmia de la primera voz pero transportandola una o varias octavas más abajo (Figura~\ref{fig:Figura2Voz3}). El número de octavas depende del tono más agudo y el tono más grave que hay en la melodía compuesta. Si se tiene que al transportar la melodía dos octavas más grave se sale del límite de representación de las notas, entonces sólo se hace más grave una octava.

		\begin{figure}[htbp]
		\centering
		\hspace*{0.0in}
		\includegraphics[scale=1]{graphics/simpletest2-F2F3_2.png}
		\includegraphics[scale=1]{graphics/simpletest3_2-BASSpartitura.png}
		\caption{Figura de entrada con los tonos producidos}
		\label{fig:Figura2Voz3}
		\end{figure}

%----------------------------------------------------------------------------------------------------------------------------------------------------

\subsection{4ª Voz: Ritmo}

Para el ritmo se creó una primera versión basada en la disposición de los vértices. Se dividía la figura en varias secciones de ángulos dejando cada vértice dentro de una sección. Estas secciones son configurables para hacerlas en distintas proporciones. Una vez se tienen los vértices de cada sección, se sustituyen por ritmos. De tal modo que si no hay un vértice en una sección, esta tiene un silencio y si hay un vértice, entonces hay un pulso de ritmo (Figura~\ref{fig:Figura1Voz4}). Este ritmo se repite durante la duración dada por entrada.

		\begin{figure}[htbp]
		\centering
		\hspace*{0.0in}
		\includegraphics[scale=0.57]{graphics/todo.png}
		\caption{Figura de entrada y los ritmos producidos}
		\label{fig:Figur1Voz4}
		\end{figure}

El otro ritmo implementado se basa en el concepto de inestabilidad de una figura de A. Pintado (\cite{portutesis}). Por defecto creamos un ritmo que consiste en pulsos largos (la duración de estos pulsos se pueden configurar). A cada nota le es asignado una serie de vértices dependiendo de la densidad de vértices de la figura. Después se va calculando cuanto de desviación se produce en esos vértices respecto al círculo de área igual a la figura y con centro el punto de equilibrio de la figura (Figura~\ref{fig:Figura2Voz4}). A mayor desviación, mayor inestabilidad luego el ritmo se subdivide en más notas y de menor duración (Figura~\ref{fig:Figura3Voz4}).

		\begin{figure}[htbp]
		\centering
		\hspace*{0.0in}
		\includegraphics[scale=1]{graphics/simpletest2-Circulo.png}
		\caption{Figura de entrada y el círculo equivalente}
		\label{fig:Figura2Voz4}
		\end{figure}

		\begin{figure}[htbp]
		\centering
		\hspace*{0.0in}
		\includegraphics[scale=1]{graphics/simpletest2-F2F3_3.png}
		\includegraphics[scale=1]{graphics/simpletest2-PERpartitura.png}
		\caption{Figura de entrada y los ritmos producidos}
		\label{fig:Figura3Voz4}
		\end{figure}

%----------------------------------------------------------------------------------------------------------------------------------------------------

\subsection{Mixer}

Los anteriores algoritmos son para crear un trozo de musica de una voz a partir de una figura, pero tenemos que saber en qué momento vamos a llamar a cada algoritmo y con qué figuras. Para ello está el Mixer.

En una primera implementación el mixer se encargaba de ordenar las figuras en orden de relevancia siguiendo el parámetro de vistosidad calculado a partir de tres características de la figura.
La primera es lo vistosidad del color de la figura. Para ello descomponemos el color en sus tres componentes principales RGB y damos peso a cada uno de ellos y lo multiplicamos por la saturación del color.
La segunda componente es el área que ocupa la figura. Cuanto más grande sea, más vistosa es.
La tercera componente es la distancia que tiene del centro de la imagen. El ojo humano enfoca de centro a laterales cuando ve una imagen por primera vez, por ello tiene más vistosidad una figura que se encuentra en el centro de la imagen que en un lateral.

	\begin{center}
		$vistosidad(Figura) =$
	\end{center}
	\begin{center}
		
		$\left\{
		\begin{array}{cc}
		A = 0.5f; B = 0.3f; C = 0.2f;\\ 
		pR = 0.45f; pG = 0.35f; pB = 0.20f;\\
		A*getSaturation()*(rgb.r*pR + rgb.g*pG + rgb.b*pB)\\
		 + B*area + C*distanceCenter
		\end{array}\right.$
	\end{center}

Tras haber ordenado las figuras por vistosidad vamos componiendo con cada una de forma iterativa las distintas voces. En el primer mixer que hicimos, sólo se componía  de melodía y ritmo. Para la segunda versión del mixer se sigue calculando la vistosidad como se explicó pero en vez de ordenar todas las figuras, ordenamos solo las figuras que están en la imagen sin estar dentro de otra figura, las llamadas figuras padre. Una vez las tenemos vamos iterando por cada figura padre componiendo su melodía, bajo y ritmo. Si la figura padre tiene otras figuras dentro de ella (figuras hijas), entonces también se añade la segunda melodía dando como entrada al algoritmo de composición de acompañamiento esta segunda figura (la figura hija). Por cada figura interior se compone un nuevo bloque de la figura padre con diferentes acompañamientos (Figura~\ref{fig:Figura1Mixer}).

		\begin{figure}[htbp]
		\centering
		\hspace*{0.0in}
		\includegraphics[scale=1]{graphics/simpletest5.png}
		\includegraphics[scale=1]{graphics/simpletest5-partitura.png}
		\caption{Figuras de entrada las voces producidas}
		\label{fig:Figura1Mixer}
		\end{figure}

Una vez tenemos compuestas todas las voces, es el mixer el que se encarga de juntarlas y llamar a la creación de la partitura y posterior conversión a midi.
