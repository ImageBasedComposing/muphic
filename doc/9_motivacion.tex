\section{Motivación}
\label{sec:motivacion}

\gotrev{Última revisión hecha el 26-06-2012}

La principal motivación de este proyecto nace, por supuesto, del interés de los participantes en la música y la aplicación de la computación a la misma. Aunque existe una lista interminable de aplicaciones orientadas a esa relación música-informática, el interés de este proyecto parte de un área en particular: la composición algorítmica.\\

El campo de la composición algorítmica consta de muchos estudios y trabajos realizados sobre la materia. Sin embargo, gran parte de ellos caen dentro de dos casos: o bien se basan en el acoplamiento y unión de diferentes piezas previamente compuestas aplicándoles ciertas modificaciones (consiguiendo resultados auditivamente agradables, pero en ningún momento ``nuevos''), o bien busca una creación completa de la pieza musical. Como ya se comentó en la sección anterior, es este último campo el que motiva el desarrollo de este proyecto. Se busca por tanto la composición genuina de piezas musicales, útil como fuente de inspiración para usuarios compositores o la generación de música de ambiente.\\

Dentro de la composición algorítmica, el interés de los integrantes del proyecto se centra sobre todo en dos aspectos fundamentales:

\begin{itemize}

	\item De todas las formas posibles de generación de música algorítmica existentes, se tiene especial interés en una generación determinista. Esto es, en vez de partir de algoritmos genéticos o cualquier otro tipo de diseño basado en un entrada aleatoria, se desea obtener una pieza musical que suponga la representación de un elemento de entrada perteneciente a contexto no auditivo. Es esta búsqueda la que lleva a plantearse el usar imágenes como entradas a estos algoritmos.
	
	\item  Se busca además que la relación entre la imagen y la música generada no esté sujeta a concepciones culturales o personales, que pueden variar en cada usuario. Es por ello que se hará uso de la sinesteria, fenómeno que relaciona diferentes sentidos, y que además es foco de muchos estudios por la rama de la psicología.
	
\end{itemize}

Se relacionan así dos elementos que incitan gran interés en la comunidad científica y que, si bien han sido estudiados por separado (como bien se aprecia en la siguiente sección), juntos componen un objeto de estudio apenas observado. Es la motivación de este proyecto el estudiar y experimentar en este ámbito, con el objetivo de expandir su trasfondo académico y observar las posibilidades que ofrece.\\

Cabe destacar también la inclinación a crear una aplicación de esta índole para dispositivos móviles. Una versión simple y accesible de este sistema puede ser de gran interés en este mercado, ya que las entradas gráficas se pueden obtener con gran facilidad gracias a las cámaras integradas en la mayoría de las plataformas portátiles. También se facilita enormemente el proceso de testeo de los diferentes resultados permitiendo su rápido progreso. Tras estudiar la viabilidad de enfocar la aplicación a sistemas móviles, se ha preferido orientar el proyecto a ordenadores personales, con el objetivo de simplicar la implementación y dar más importancia al diseño de algoritmos de composición. No obstante, se ha preparado la aplicación para facilitar una futura portabilidad a móviles.\\