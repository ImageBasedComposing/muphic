\section{Módulo de análisis}

\todo{referenciar figuras,	analizer, uso de opencv como paquete}


El módulo de análisis, como ya se ha comentado, es el encargado de, dada una imagen dada y una configuración de entrada, producir un archivo XML con la lista de polígonos que componen la imagen. Esta lista debe componer una nueva imagen lo más fiel posible a la imagen de entrada, teniendo en cuenta los ajustes especificados en el archivo de configuración.\\

Este módulo hace un gran uso de la librería OpenCV, de la cual se ayuda tanto para tratar imágenes como para componer formas y polígonos.\\

La estructura general del módulo se ve en la Figura~\ref{fig:diagramageneralPHIC}.\\

		\begin{figure}[htbp]
		\centering
		\includegraphics[scale=0.47]{graphics/todo.png}
		\caption{Vista general del módulo de anális}
		\label{fig:diagramageneralPHIC}
		\end{figure}
	
La vista clase por clase se ve en la Figura~\ref{fig:diagramaclasesPHIC}.\\

		\begin{figure}[htbp]
		\centering
		\includegraphics[scale=0.47]{graphics/todo.png}
		\caption{Vista de las clases del módulo de anális}
		\label{fig:diagramaclasesPHIC}
		\end{figure}
		
Y el flujo de ejecución se ve en la Figura~\ref{fig:diagramaflujoPHIC}.\\

		\begin{figure}[htbp]
		\centering
		\includegraphics[scale=0.47]{graphics/todo.png}
		\caption{Flujo de ejecución del módulo de anális}
		\label{fig:diagramaflujoPHIC}
		\end{figure}
		
Una vista final del módulo como paquetes se ve en la Figura~\ref{fig:diagramapaquetesPHIC}.\\

		\begin{figure}[htbp]
		\centering
		\includegraphics[scale=0.47]{graphics/todo.png}
		\caption{Diagrama de paquetes del módulo de anális}
		\label{fig:diagramapaquetesPHIC}
		\end{figure}