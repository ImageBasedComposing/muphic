\section{Introducción}


	\subsection{¿Qué es Muphic?}
	
		\vspace{0.2in}
		Muphic es un software capaz de producir piezas musicales compuestas automáticamente a partir de una entrada gráfica estática. Es el resultado de un proyecto de investigación y desarrollo comenzado en octubre del 2011 por 3 alumnos de Ingeniería Informática, cuyo propósito es obtener una correlación entre música e imágenes.\\
		
		De forma resumida, dado que se entrará en detalles en secciones siguientes, la aplicación analiza una imagen de entrada y utiliza el contenido de dicho análisis para generar una pieza musical mediante algoritmos de composición automática.\\
		
		Este proyecto trata por tanto 3 temas distintos:
		
		\begin{itemize}
		
		\item Análisis de imágenes: la imagen de entrada se debe procesar y estudiar hasta obtener la información deseada.
		\item Correlación entre música e imágenes: Los datos obtenidos se interpretan y relacionan con notas y ritmos musicales.
		\item Composición algoritmica: Con la información obtenida a partir de la imagen y el conocimiento de cómo se relaciona con la música, se pasa a componer, de forma automática, una pieza musical completa.
		\end{itemize}
		
		El tercer punto va a depender enormemente del segundo, ya que la forma distintas correlaciones música-imágen van a conllevar diferentes estructuras y diseños de algoritmos de composición.\\
		
		En el desarrollo de este proyecto hay ciertos detalles importantes que se han considerado y es necesario mencionar.
		
		\todo{esto es más rollo "subtitulo: *nueva linea* contenido indentado un poco" que un itemize o una subsubsection, pero no sabía ponerlo. De todas formas algunos de estos son requisistos no funcionales y asi ya tenemos apartado nuevo para la parte de sistema :)}
		
		
		En el ámbito del análisis gráfico, tenemos que:
		\begin{itemize}
		
		\item \underline{No se tienen en cuenta objetos físicos}:\\
			El objetivo del análisis no es tanto obtener información de qué es lo que la imagen representa, si no los colores y formas que contiene, y su distribución y características dentro de la misma.
		\end{itemize}
		
		En el ámbio de la composición y correlación imagen-música:
		
		\begin{itemize}
		
		\item \underline{El único componente de entrada usado es la imagen dada}:\\
			El objetivo de este proyecto es obtener piezas musicales en base a imágenes, por lo que, en lo que a composición se refiere, no se usará más información que la proporcionada por la imagen. Esta caracteríca impide usar ningún tipo de pista musical externa, base de datos o cualquier tipo de entrada o ayuda externa en el proceso de composición.
		\item \underline{La imagen es estática}:\\
			Se consideran para el análisis formatos de imagen estática (tales como bmp, jpg o png), y no animada (video o archivos de animación). Este planteamiento condiciona el proceso de correlación imágen-música, ya que se pretende obtener una salida no estática, como es la música, a partir de una imágen inmóvil. Veremos más adelante cómo se obtiene ese efecto de \emph{dinamismo} a partir de la información proporcionada por una imágen.
		\item \underline{La base de la correlación será la sinestesia}:\\	
			La sinestesia, que se explica en posteriores secciones, es la pieza clave en la creación de música basada en imágenes, y no la reacción psicológica del cerebro humano ante los colores. Esto es, concepciones tales como \emph{el azul es un color frío} o \emph{el rojo produce apetito} no se tienen en cuenta.\\
			\todo{Corrección: el esto es me suena fatal, pondría algo así como un ..., es decir ..., aparte el rojo apetito parece una coña, pondría algo mas como el rojo transmite agresividad (por lo del rollo formal y eso)}
			\todo{poner referencia a estas relaciones, que las hay}
		\item \underline{Generación de contenido frente a acoplamiento de creaciones preestablecidas}:\\				
			Es importante recalcar que, en el proceso de composición, como ya se adelantaba en el primer punto, se da especial importancia a que las piezas generadas sean generadas de cero y no partan de ninguna estructura predefinida. No se usarán por tanto piezas ya compuestas o estructuras conocidas como \emph{ladrillos} para construír una pieza nueva.
		\item \underline{Compromiso entre \emph{sonar bien} y \emph{ser interesante}}:\\
			Durante todo el proyecto se ha intentado abarcar bien ambos conceptos, deseando que las piezas compuestas no sean en ningún momento desagradables de escuchar, pero permitiendo cierta libertad para que los algoritmos puedan producir resultados frescos e interesantes.
		\item \underline{Se trata de un campo muy estudiado pero a la vez muy \emph{verde}}:\\	
			Aunque existen gran cantidad de estudios sobre la sinestesia y la composición musical en base a imagenes, es un ámbito muy abierto y en el que todavía queda mucho por estudiar. Más adelante se explicará con detalle este aspecto.
			
		\end{itemize}
			
		Teniendo eso en cuenta, comentaremos en las siguientes secciones cómo se ha abordado el proceso. 
	

\section{Motivación}
\label{sec:motivacion}

\torev{PENDIENTE DE PRIMERA REVISIÓN}\\

La principal motivación de este proyecto nace, por supuesto, del interés de los participantes en la música \color{blue}y la aplicación de la computación a la misma\color{black}. Aunque existe una lista interminable de aplicaciones orientadas a esa relación música-informática, el interés de este proyecto parte de un área en particular: la composición algorítmica.\\

El campo de la composición algorítmica consta de muchos estudios y trabajos realizados sobre la materia. Sin embargo, gran parte de ellos caen dentro de dos casos: o bien se basan en el acoplamiento y unión de diferentes piezas previamente compuestas aplicándoles ciertas modificaciones (consiguiendo resultados auditivamente agradables, pero en ningún momento ``nuevos''), o bien busca una creación completa de la pieza musical. Como ya se comentó en la sección anterior, es este último campo el que motiva el desarrollo de este proyecto. Se busca por tanto la composición genuina de piezas musicales, útil como fuente de inspiración para usuarios compositores o la generación de música de ambiente.\\

Dentro de la composición algorítmica, el interés de los integrantes del proyecto se centra sobre todo en dos aspectos fundamentales:

\begin{itemize}

	\item De todas las formas posibles de generación de música algorítmica existentes, se tiene especial interés en una generación determinista. Esto es, en vez de partir de algoritmos genéticos o cualquier otro tipo de diseño basado en un entrada aleatoria, se desea obtener una pieza musical que suponga \color{blue}la representación de un elemento de entrada perteneciente a contexto no auditivo\color{black}. Es esta búsqueda la que lleva a plantearse el usar imágenes como entradas a estos algoritmos.
	
	\item  \color{blue}Se busca además que la relación entre la imagen y la música generada no esté sujeta a concepciones culturales o personales, que pueden variar en cada usuario. Es por ello que se hará uso de la sinesteria, fenómeno que relaciona diferentes sentidos, y que además es foco de muchos estudios por la rama de la psicología.\color{black}
	
\end{itemize}

Se relacionan así dos elementos que incitan gran interés en la comunidad científica y que, si bien han sido estudiados por separado (como bien se aprecia en la siguiente sección), juntos componen un objeto de estudio apenas observado. Es la motivación de este proyecto el estudiar y experimentar en este ámbito, con el objetivo de expandir su trasfondo académico y observar las posibilidades que ofrece.\\

Cabe destacar también la inclinación a crear una aplicación de esta índole para dispositivos móviles. Una versión simple y accesible de este sistema puede ser de gran interés en este mercado, ya que las entradas gráficas se pueden obtener con gran facilidad gracias a las cámaras integradas en la mayoría de las plataformas portátiles. También se facilita enormemente el proceso de testeo de los diferentes resultados permitiendo su rápido progreso. \color{blue}Tras estudiar la viabilidad de enfocar la aplicación a sistemas móviles, se ha preferido orientar el proyecto a ordenadores personales, con el objetivo de simplicar la implementación y dar más importancia al diseño de algoritmos de composición. No obstante, se ha preparado la aplicación para facilitar una futura portabilidad a móviles.\\\color{black}
\section{Estado del Arte}
\label{sec:estadodelarte}

\subsubsection{Estudio sobre Sinestesia}

Dentro de la sinestesia, la parte que es importante en este proyecto es la llamada sinestesia musical. Aunque no es el objetivo realizar una investigación rigurosa y llegar a una conclusión o afirmar una suposición realizada con antelación, estudiamos la sinestesia porque cabe pensar que sería una aproximación muy buena para poder empezar a realizar la conversión de imagen a música.
Según \emph{(Video Redes:Flipar en Colores)}David Eagleman las personas de forma innata  tenemos la tendencia de conectar sentidos, entre otros visual y auditivo (sonidos con formas y sonidos con colores).\\

Varios artistas en el tiempo han expresado su capacidad sinestésica a través de sus obras o documentando y experimentando con este fenómeno. Entre ellos se encuentra Scriabin, quien hizo una asociación entre los colores y los acordes que aparecen en música de forma rigurosa. También hay científicos como Newton que intentaron asignar colores a la notas pero no es un problema reciente, ya desde la antigua Grecia con Aristóteles y Pitágoras se lleva intentando dar una solución definitiva. Después de ver la aproximación a través de la experiencia o a través de la lógica o fundamentación matemática queda claro que cada persona sinestésica es única y puede haber grandes diferencias entre unas experiencias sinestésicas y otras.\\

Las investigaciones llevadas a cabo por el psicólogo Wolfgang Köhler en 1929 y las hechas recientemente demuestran que tenemos una conexión profunda entre los sonidos y las formas de lo que vemos. 

\todo{creo mejor no hablar de esto:Su experimento se basaba en dejar asignar dos palabras (sonidos) a dos figuras...kiki-booma}

 Es por tanto importante poder encontrar esa asociación y ponerla en práctica a la hora de convertir las imágenes en música. Lawrence E. Marks, The Unity of the Senses encuentra una asociación entre los tiempos en música y las formas, que cuanto más angular e irregular es una figura el tempo o las notas son más rápidas.También declara sobre los colores que \emph{``Sadly, one finds that synesthetic associations between colors and musical notes fail to favor any particular scheme over the others.''}: Por desgracia, al final uno descubre que no hay una asociación sinestésica entre notas musicales y colores que prevalezca ante las demás.\\

Kandinsky, otro artista como Scriabin pero partiendo de la pintura, ha investigado y defendido fervientemente la posibilidad de asociar la música a la pintura.De varios de sus estudios y siguiendo sus propias experiencias sinestésicas ha escrito abundante documentación. Relaciona otros aspectos como por ejemplo que la escala de intensidad de los sonidos estaba relacionada con la intensidad de los trazos de la pintura, también distingue timbres de instrumentos por colores asignando a cada familia colores parecidos. \\


\subsection{Composición basada en imagenes}

\torev{Última revisión realizada: 5-06-2012}

Para poder componer música a partir de datos gráficos necesitamos crear una correlación entre los elementos que tenemos en la parte gráfica y en la parte musical. Esta correspondencia no es fácil de conseguir, y a través de la historia se han desarrollado varias teorías que se han puesto en práctica. Los primeros intentos de síntesis fueron los instrumentos llamados ``órganos de color'' que mostraban acompañando al sonido una muestra visual.

El primer instrumento que se construyó fue por parte de Louis Bertrand Castel (1730) y se trataba de un clavecín al que le incorporaron una pantalla encima y un sistema de iluminación. Cuando se apretaban las teclas se iluminaban los colores correspondientes en la pantalla (\cite{organosColor}). \color{blue} A este experimento le continuaron muchos otros, pero al ser una correspondencia muy directa entre nota y color da poca diversidad. Además es una correspondencia de música a la estimulación visual mientras que nuestro interés se centra en la otra dirección, de la imagen a la música.\\ \color{black}

Desde hace unas décadas, gracias a los avances tecnológicos, se ha investigado activamente en algoritmos automáticos de composición. Una posible clasificación de los diferentes algoritmos basada en su característica principal sería: modelos matemáticos, sistemas basados en conocimiento, gramáticas, evolutivos, sistemas con aprendizaje e híbridos (\cite{AIMethodsForComposition}). \\
\color{blue} En los algoritmos matemáticos destacamos Stochastic (Markov chains) y Chaos, su principal inconveniente es representar a alto nivel detalles más generales o abstractos de la música. En los sistemas basados en conocimiento como CHORAL o el armonizador de Pachet and Roy tienen la dificultad de representar todo el conocimiento musical necesario, que en este dominio es muy grande. En las gramáticas se tiene como ejemplo el proyecto EMI (Experiments in Musical Intelligence) o Steedman y su generador de música Jazz. El mayor problema que presentan las gramáticas es la rigidez de las mismas ya que la música es ambigua y con muchas excepciones a las reglas. Los algoritmos evolutivos su principal problema es la función de evaluación o fitness, tenemos McIntyre que usa una función de evaluación automática y la herramienta de improvisación de jazz de Biles que usa una evaluación humana. En los sistemas con aprendizaje su mayor inconveniente es que estos sistemas aprenden sólo lo básico de la música compuesta y no de los niveles más altos de composición, destacamos EBM y MUSE. Los sistemas híbridos intentan combinar lo mejor de los diferentes sistemas posibles, dentro de esta clase encontramos HARMONET que combina sistemas con aprendizaje y basados en conocimiento.
\color{black}

Pero no solo la algoritmia es variada, también podemos clasificar de diferente forma según el tipo de música que se quiera componer: micro-composición (diseño de sonidos) y la macro-composición (combinación de sonidos ya diseñados) (\cite{AudioVisualSurvey}). \color{blue} En nuestro caso queremos separar los algoritmos de macro-composición en: basados en el fenómeno de la sinestesia y los que no siguen esta línea.\\ \color{black}

Hay varias aproximaciones de compositores basados en imagen , algunos de ellos como Phonogramme (\cite{ImageBaseComposition} \cite{Phonogramme}) que es un editor gráfico, interpreta la imagen como una partitura. \color{blue} La imagen representa la relación bidimensional de tonos y duración de los sonidos, es decir, la imagen es una partitura que se lee de izquierda a derecha en el tiempo y de abajo a arriba en la altura de los tonos. El problema de este algoritmo desde el punto de vista de la sinestesia es que las imágenes son partituras y deben estar diseñadas para ser usadas para este fin (usando el editor gráfico), no acepta cualquier entrada gráfica. Esta línea de investigación no sigue o intenta imitar la sinestesia.\\ \color{black}

Otra opción interesante es la propuesta basada en el concepto de ``croma'' (o índice de cromatismo). Se subdivide la imagen en bloques (ladrillos cromáticos) los cuales tienen un índice de cromatismo y sirve para generar o asignar trozos de música (\cite{bricksConvertsMusic}). Estos trozos de música pueden cogerse de una base de datos o ser creados por un experto (compositor). \color{blue} Al no haber una correspondencia directa entre la imagen y la música, este proyecto tampoco sigue la línea de la sinestesia.\\ \color{black}

También destacamos el trabajo realizado por Xiaoying Wu y Ze-Nian Li (\cite{ImageBaseComposition}) en el que analizan la imagen en tres pasos. Primero \emph{Hacer una partición} de la imagen en piezas (divisiones, trozos) más pequeñas. Segundo realizan la \emph{Secuenciación} de esas piezas para darles un orden en el tiempo. Por último aplican un \emph{Mapeado} de las piezas de imagen en notas musicales. \color{blue} Al ser una correspondencia directa entre la imagen y la música, este trabajo sigue parcialmente la sinestesia y la psicología de las formas y colores aunque no es su propósito central.\\ \color{black}

\color{blue}
Por último destacamos el trabajo de A. Pintado (\cite{portutesis}) en el que hace una investigación de la sinestesia y la percepción de las formas para poder generar ritmos. Analizando una imagen de entrada sacando sus formas y líneas hace una relación directa con diferentes ritmos que genera. Este trabajo se centra en la línea de la sinestesia y por tanto ha sido una fuente de información importante. \color{black}

Todos estos algoritmos tienen en común un problema, que está implícito, y es que dada una imagen cada persona espera una música diferente como correspondencia. Por tanto es difícil determinar el fitness o validez de cada algoritmo sabiendo además que la entrada gráfica puede tener infinidad de valores posibles.


