\chapter{Introducción}
\label{chap:intro}


\gotrev{Última revisión hecha el 13-06-2012}\\

	\section{¿Qué es Muphic?}
	\label{sec:quees}
	
		\vspace{0.2in}
		Muphic es un software capaz de producir piezas musicales compuestas automáticamente a partir de una imagen. Es el resultado de un proyecto de sistemas informáticos comenzado en octubre del 2011 por 3 alumnos de Ingeniería Informática, cuyo propósito es generar música a partir de imágenes basándose en el fenómeno de la sinestesia.\\
		
		Antes de proceder a describir el proyecto, es necesario explicar qué es la sinestesia, un término de gran importancia a lo largo del documento. Tal y como afirman Ramachandran y Hubbard en \cite[pg.~4]{paperSyn}:
		
		\begin{quote}
		\emph{``La Sinestesia es una curiosa condición según la cual un individuo de cualquier otra manera normal experimenta sensaciones en una modalidad cuando una segunda modalidad es estimulada. Por ejemplo, un sinésteta puede experimentar un color determinado siempre que se encuentre con un tono particular(p. ej., C\# puede ser azul) o puede ver un número dado tintado siempre de un cierto color (p. ej., ‘5’ puede ser verde y ‘6’ puede ser rojo).''}
		\end{quote}
		
		Dentro de la sinestesia, nos centraremos en la sinestesia auditivo-visual (aquella que relaciona los sentidos del oído y la vista), y más concretamente en la estimulación del sentido auditivo a partir de una percepción visual. Estudiaremos cómo reproducir esa asociación neurológica de las sensaciones visuales con los fragmentos musicales para poder generar composiciones musicales.\\
		 
		
		De forma resumida, dado que se entrará en detalle en secciones siguientes, la aplicación desarrollada analiza una imagen de entrada y utiliza el contenido de dicho análisis para generar una pieza musical mediante algoritmos de composición automática.\\

		El compositor trabaja en dos etapas secueciadas:
		
		\begin{itemize}
		
		\item Análisis de imágenes: la imagen de entrada se debe procesar y estudiar hasta obtener la información deseada (formas, colores, tamaños, ...).

		\item Composición algorítmica: Con la información obtenida a partir de la imagen y el conocimiento de cómo se relaciona con la música, se pasa a componer, de forma automática, todas las partes que forman nuestra pieza musical: ritmo, melodía y armonía.
		\end{itemize}
		
		Naturalmente, la segunda etapa va a depender del resultado del análisis de la primera: con distintos análisis de imagen se obtiene pequeñas diferencias en las descripciones de las imágenes de entrada. Y por tanto las composiciones musicales serán parcialmente diferentes, ya que la información procesada por el compositor será distinta en cada caso. Además, la elección de los algoritmos de composición cambiará de forma significativa la pieza musical final.\\
		
		En el desarrollo de este proyecto hay ciertos aspectos importantes que se ha considerado y es necesario mencionar:	

		\begin{itemize}
		
		\item \emph{La base de la correlación audio-musical será la sinestesia}:
			\vspace{0.1in}
			\\La sinestesia, como ya se ha comentado, es la pieza clave en la creación de música basada en imágenes (Sección~\ref{subsubsec:estudioSinestesia}) frente a otras alternativas de planteamiento.\\
			Entre estas alternativas, cabe destacar la reacción psicológica sociocultural del cerebro humano ante la música y los colores. Es decir, estudios muestran que el verde, el azul y otros colores con longitudes de onda bajas se relacionan con la calma, mientras que colores con longitudes de onda altas aumentan el nerviosismo y la inestabilidad \cite{colorpsy}. Hay, por otro lado, multitud de investigaciones sobre la psicología de la música y su relación con distintas emociones y sensaciones, como las asociaciones entre los modos griegos y los estados de ánimo \cite{micrologus}. Una posible rama de desarrollo procedería juntando ambos ámbitos, la psicología del color y la de la música, para generar la música deseada en función de la imagen dada.\\
		\item \emph{No se tienen en cuenta objetos físicos}:
			\vspace{0.1in}
			\\El objetivo del análisis no es tanto obtener información de qué es lo que la imagen representa (reconocimiento e interpretación), sino los colores y formas que contiene, y su distribución y características dentro de la misma (segmentación y descripción). Es decir, si tenemos una imagen con un elefante no queremos reconocer el ``elefante'' sino que hay una figura  redondeada con una parte alargada (la trompa) y que tiene un colo azul grisáceo y se encuentra en el centro de la imagen.
		\item \emph{La imagen es estática}:
			\vspace{0.1in}
			\\Se consideran para el análisis formatos de imagen estática (tales como bmp, jpg o png), y no animada (vídeo o archivos de animación). Este planteamiento condiciona el proceso de correlación imagen-música, ya que se pretende obtener una salida no estática, como es la música, a partir de una imagen inmóvil. Veremos más adelante cómo se obtiene ese efecto de \emph{dinamismo} a partir de la información proporcionada por una imagen.
		\item \emph{Generación de contenido frente a acoplamiento de creaciones preestablecidas}:
			\vspace{0.1in}
			\\Es importante recalcar que en el proceso de composición las piezas generadas se construyen desde cero y no parten de ninguna estructura predefinida. No se usarán por tanto piezas ya compuestas o estructuras conocidas como ``ladrillos'' para construir una pieza nueva, como una base de datos o adaptaciones de piezas musicales enteras. Es cierto que se pueden seleccionar ciertos parámetros de la composición pero sirven para matizar las composiciones (tempo, instrumentos, ...).
		\end{itemize}
		
		Además, resulta importante resaltar que siendo el objetivo final del proyecto generar música no se ha buscado que compita con obras de grandes compositores. Nuestro modelo es la creación de la llamada música de ambiente; es decir, música que, siendo voluntariamente no atrayente ni excesivamente interesante, tiene como requisito principal el no ser molesta. Se sigue por tanto la línea de la música mostrada por Brian Eno en \emph{Música para Aeropuertos (1978)}:
		
		\begin{quote}
		\emph{``[La música ambiente es] Algo de lo que puedes entrar o salir discretamente. Puedes atender o puedes elegir no distraerte con ella si quisieras hacer algo mientras la música está reproduciéndose.''}, (\cite{BrianEnoInterview}, entrevista con Brian Eno).
		\end{quote}		
		
		Por último, se ha de añadir que el ámbito de la composición basada en imágenes es uno muy estudiado pero a la vez poco desarrollado. Es decir, aunque existe una gran cantidad de estudios sobre la sinestesia y la composición musical automática (como se puede ver en la Sección~\ref{sec:estadodelarte}), aún quedan muchas cuestiones que investigar y áreas que profundizar así como la conexión entre la música y la imagen.\\

\section{Motivación}


La principal motivación de este proyecto nace, por supuesto, del interés de los participantes en la música y su aplicación en el área de la computación. Aunque existe una lista interminable de aplicaciones orientadas a esa relación música-informática (como puede ser SunVox \cite{SunVox} en el ámbito de los sintetizadores musicales, o la famosa aplicación ToneMatrix \cite{toneMatrix}, por citar dos ejemplos), el interés de este proyecto parte de un área en particular: la composición algorítmica.\\

El campo de la composición algorítmica consta de muchos estudios y trabajos realizados sobre la materia. Sin embargo, gan parte de ellas caen dentro de dos casos: o bien se basan en el acoplamiento y unión de diferentes piezas previamente compuestas aplicándoles ciertas modificaciones (consiguiendo resultados auditivamente agradables, pero en ningún momento ``nuevos''), o bien busca una creación completa de la melodía. Como ya se comentó en la sección anterior, es este último campo el que motiva el desarrollo de este proyecto. Se busca por tanto la composición genuína de piezas musicales, útil como fuente de inspiración para usuarios compositores o la generación de música de ambiente.\\

Dentro de la composición algorítmica, el interés de los intregrantes del proyecto se centra sobre todo en dos aspectos fundamentales:

\begin{itemize}

	\item De todas las formas posibles de generación de música algoritmica existentes, se tiene especial interés en una generación determinista. Esto es, en vez de partir de algortimos genéticos o cualquier otro tipo de diseño basado en un entrada aleatoria, se desea obtener una pieza musical que suponga la representación de un objeto constante y perteneciente a conexto no auditivo. Es esta búsqueda la que lleva a plantearse el usar imagenes como entradas a estos algoritmos.
	
	\item Además, dado que se tiene una entrada gráfica al algoritmo, la interpretación de la misma no esté sujeta a concepciones, ya sean culturales o personales, de nosotros que diseñamos los algoritmos de composición. Buscando cumplir este objetivo es donde nos encuentramos con la sinestesia, fenómeno que relaciona diferentes sentidos que además está siendo fuertemente estudiada por la rama de la psicología.
	
\end{itemize}

Se relacionan así dos elementos que incitan gran interés en la comunidad científica y que, si bien han sido estudiados por separado (como bien se aprecia en la siguiente sección), juntos componen un objeto de estudio apenas observado. Es la motivación de este proyecto el estudiar y experimentar en este ámbito, con el objetivo de expandir su trasfondo académico y observar las posibilidades que ofrece.\\

Cabe destacar también la inclinación a crear una aplicación de esta índole para dispositivos móviles. Una versión simple y accesible de este sistema puede ser de gran interés en este mercado, ya que las entradas gráficas se pueden obtener con gran facilidad gracias a las cámaras integradas en la mayoría de las plataformas portátiles. También se facilita enormemente el proceso de testeo de los diferentes resultados permitiendo su rápido progreso.
\section{Estado del Arte}
\label{sec:estadodelarte}

\subsection{Estudio sobre Sinestesia}
\label{subsubsec:estudioSinestesia}

\todo{
Esquema:
\begin{itemize}
\item introducción sinestesia color-música
\item los primeros que asociaron color con sonido
\item una aproximación a través de la psicología
\item experimentos y cosas hechas sobre la sinestesia (sin técnicas de composición)
\end{itemize}
}


Dentro de la sinestesia, la parte que es importante en este proyecto es la llamada sinestesia musical. Esta consiste en mezclar la experiencia auditiva con la visual. Se ha estudiado este tipo de sinestesia porque cabe pensar que sería una fuente de información razonable para poder empezar a realizar la conversión de imagen a música.
Según el neurocientífico David Eagleman, las personas de forma innata tienen la tendencia de conectar sentidos, entre otros visual y auditivo (sonidos con formas y sonidos con colores, como se ve en \cite{VideoRedesFliparColores}).\\

Varios artistas y científicos en el tiempo han expresado su capacidad sinestésica a través de sus obras o documentando y experimentando con este fenómeno. Desde la antigua Grecia se intentaba encontrar esa equivalencia entre los colores y los sonidos. Aristóteles en su ensayo (De Sensu et Sensato \cite{DeSensuEtSensato}, 439b30) realiza una descripción de los colores comparándolos directamente con la armonía presente en la música. Siguiendo esa línea científica-filosófica, Newton asignó un color a cada nota de la escala diatónica siguiendo los colores de su prisma de C (Do) rojo a B (Si) violeta.\\

Otra aproximación a la relación música-imagen que normalmente siguen los artistas es a través de la experiencia con el fenómeno de la sinestesia. Encontramos entre ellos al compositor Scriabin, quien hizo una asociación entre los colores y los acordes que aparecen en música usando el círculo de quintas. Kandinsky, otro artista como Scriabin pero pintor, partiendo de la pintura ha investigado y defendido fervientemente la posibilidad de asociar la música a la pintura y su relación directa (\cite{ConcerningSpiritualArt}). Resalta aspectos como por ejemplo que la escala de intensidad de los sonidos estaba relacionada con la intensidad de los trazos de la pintura, también distingue timbres de instrumentos por colores asignando a cada familia colores parecidos.\\

Las investigaciones llevadas a cabo por el psicólogo Köhler en 1929 (\cite{GestaltPsychology}) y las hechas recientemente demuestran que tenemos una conexión profunda entre los sonidos y formas, además de los colores. Marks, en \emph{The Unity of the Senses} (\cite{TheUnityOfTheSenses}) encuentra una asociación entre los tiempos en música y las formas, tal que cuanto más angular e irregular es una figura el tempo o las notas son más rápidas.También declara sobre los colores que:
\begin{quote}
\emph{``Por desgracia, al final uno descubre que no hay una asociación sinestésica entre notas musicales y colores que prevalezca ante las demás''}.\\
\end{quote}

Después de ver la aproximación a través de la experiencia (con los trabajos de Aristótles, Scriabin o Kardinsky ya mencionados) a través de la lógica o fundamentación matemática (como los propuestos por Newton o Köhler), queda claro que cada persona sinestésica es única y puede haber grandes diferencias entre unas experiencias sinestésicas y otras. Es por tanto importante poder encontrar asociaciones aceptables y ponerlas en práctica para comprobar los distintos resultados.

\subsection{Composición basada en imagenes}

\torev{Última revisión realizada: 20-06-2012}\\

Para poder componer música a partir de datos gráficos necesitamos crear una correlación entre los elementos que tenemos en la parte gráfica y en la parte musical. Esta correspondencia no es fácil de conseguir, y a través de la historia se han desarrollado varias teorías que se han llegado a poner en práctica. Los primeros intentos de síntesis fueron los instrumentos llamados ``órganos de color'' que acompañaban al sonido con una muestra visual.

El primer instrumento lo construyó Louis Bertrand Castel (1730), se trataba de un clavecín al que se le había incorporado una pantalla y un sistema de iluminación. Cuando se pulsaban las teclas se iluminaban los colores correspondientes en la pantalla \cite{organosColor}. A este experimento le continuaron muchos otros que incorporaban mejoras, pero al ser una correspondencia muy directa entre nota y color, al final se obtiene poca diversidad. Además es una correspondencia de la música a la estimulación visual mientras que nuestro interés se centra en la otra dirección, de la imagen a la música.\\

Desde hace unas décadas, gracias a los avances tecnológicos, se han investigado activamente los algoritmos automáticos de composición. Una posible clasificación de los diferentes algoritmos basada en su característica principal sería: modelos matemáticos, sistemas basados en conocimiento, gramáticas, evolutivos, sistemas con aprendizaje e híbridos \cite{AIMethodsForComposition}. \\

\color{blue}
\begin{itemize}
	\item Modelos matemáticos: los más usados son los procesos basados en sistemas estocásticos y cadenas de Markov. Como ejemplo significativo de estos modelos destaca Cybernetic Composer \cite{AIMusicSurvey}. Otra subcorriente son los basados en la teoría del caos \cite{ChaosTeoriaMusica}.
	\item Sistemas basados en conocimiento: dependiendo de cómo se represente el conocimiento y cómo se manipula podemos hacer diferentes clasificaciones. Como ejemplos relevantes se tienen CHORAL \cite{HistoryAlgorithmicComp} o SICOM \cite{SICOM}.
	\item Gramáticas: fueron las primeras técnicas usadas. Se suelen mezclar con técnicas probabilísticas obteniendo gramáticas indeterministas, ya que si no, puede producirse música poco variada. Destacan el proyecto EMI \cite{HistoryAlgorithmicComp} o Steedman y su generador de música Jazz \cite{AIMethodsForComposition}.
	\item Algoritmos evolutivos: se dividen en dos posibilidades según la función de evaluación. La primera es usando una función de evaluación automática. Como ejemplo se tiene McIntyre \cite{AIMethodsForComposition}. La otra posibilidad es usar una evaluación humana, que es bastante más lenta y además ambigua. La herramienta de improvisación de Jazz de Biles, GenJam \cite{GenJam}, es un ejemplo importante de este tipo.
	\item Sistemas con aprendizaje: están abiertas varias líneas de investigación según las diferentes formas del proceso de aprendizaje (adquisición de conocimiento del sistema). Una posibilidad es través de redes neuronales artificiales, un ejemplo significativo es EBM \cite{AIMethodsForComposition}. Otra manera es con aprendizaje automático (aprendizaje máquina) donde destaca el ejemplo de MUSE \cite{AIMethodsForComposition}.
	\item Híbridos: intentan combinar lo mejor que ofrecen los diferentes sistemas posibles, un ejemplo relevante es HARMONET \cite{AIMethodsForComposition} que combina sistemas con aprendizaje (redes neuronales) con sistemas basados en conocimiento.\\
\end{itemize}
\color{black}

Pero no solo la algoritmia es variada, también podemos clasificarla de diferente forma según el tipo de música que se quiera componer: micro-composición (diseño de sonidos) y la macro-composición (combinación de sonidos ya diseñados para la creación de una obra musical) \cite{AudioVisualSurvey}. En nuestro caso nos interesan los algoritmos de macro-composición basados en el fenómeno de la sinestesia.\\ 

Hay varias aproximaciones de compositores basados en imagen , algunos de ellos como Phonogramme \cite{ImageBaseComposition} \cite{Phonogramme} consisten en un editor gráfico que interpreta la imagen como una partitura. La imagen representa la relación bidimensional de tonos y duración de los sonidos, es decir, la imagen es una partitura que se lee de izquierda a derecha en el tiempo y de abajo a arriba en la altura de los tonos. 
\\El problema de este algoritmo desde el punto de vista de la sinestesia es que las imágenes son partituras y deben estar diseñadas para ser usadas con este fin (usando el editor gráfico), no acepta cualquier entrada gráfica. Esta línea de investigación se hace valer de la consexión psicológica que tenemos entre formas y sonidos, pero realmente no busca la sinestesia como base.\\

Otra opción es la propuesta basada en el concepto de ``croma'' (o índice de cromatismo) \cite{bricksConvertsMusic}. Se subdivide la imagen en bloques (ladrillos cromáticos) los cuales tienen un índice de cromatismo, esto sirve para generar o asignar trozos de música que pueden ser cogidos de una base de datos o ser creados por un experto (compositor). Al no haber una correspondencia directa entre la imagen y la música, este proyecto no está fundamentado en la sinestesia. Si bien recoge algunas ideas más adelante pierde la esencia de la sinestesia.\\

También destacamos el trabajo realizado por Xiaoying Wu y Ze-Nian Li \cite{ImageBaseComposition} en el que se analiza la imagen en tres pasos. Primero: hacer una \emph{Partición} de la imagen en piezas (divisiones, trozos) más pequeñas. Segundo: realizar la \emph{Secuenciación} de esas piezas para darles un orden en el tiempo. Tercero: aplicar un \emph{Mapeado} de las piezas de imagen a notas musicales.
\\Al ser una correspondencia directa entre la imagen y la música, este trabajo sigue parcialmente la sinestesia y la psicología de las formas y colores aunque este no es su propósito final.\\ 

\color{blue}De forma adicional, cabe mencionar dos trabajos interesantes en un ámbito menos académico.
El primero es un trabajo realizado por un equipo ruso \cite{dibujosymusica}, que permite hacer un dibujo simple (compuesto por un único trazo) y observar cómo se transforma en una melodía distinta dependiendo del trazo realizado y la herramienta con la que se ha realizado. Aunque parte de una base musical estática, a la que van ``maquillando'' de distintas formas dependiendo de la entrada gráfica, es digno de mención su facilidad de uso y la calidad de los resultados obtenidos.\\

El segundo de ellos es el realizado por el físico Lauri Gröhn \cite{rusofotos}, que aplica una serie de reglas basadas en la sinestesia para realizar composiciones algorítmicas postprocesadas. Basa el análisis en un estudio por secciones de la figura de entrada, mediante el cual va generando pequeñas porciones musicales a medida que recorre distintas regiones de la imagen divididas previamente de forma uniforme.\\\color{black}

Por último destacamos el trabajo de A. Pintado \cite{portutesis} en el que hace una investigación de la sinestesia y la percepción de las formas para poder generar ritmos. Analizando una imagen de entrada obtiene sus formas y líneas los cuales traduce a ritmos que genera usando  una relación directa entre el ritmo y la inestabilidad de las figuras. Este trabajo se centra en la línea de la sinestesia, especialmente en las formas de las figuras y los ritmos musicales, dejando de lado los colores y los tonos musicales. Por tanto ha sido una buena fuente de información aunque trate sólo parcialmente nuestro objetivo.\\ 

Todos estos algoritmos tienen en común un problema implícito puesto que dada una imagen cada persona espera una correspondencia musical diferente. Por tanto es difícil determinar el fitness o validez de cada algoritmo sabiendo además que la entrada gráfica puede tener infinidad de interpretaciones de todos sus valores disponibles.



\section{Visión general del documento}
\torev{Pendiente de primera revisión}

El presente documento esta estructurado del siguiente modo:\\

En primer lugar, se ha realizado una primera toma de contacto del proyecto y su contexto en el Capítulo~\ref{chap:intro}, \textit{\nameref{chap:intro}}. Este ha comenzado con una descripción del sistema a modo de introducción en la Sección~\ref{sec:quees}, \textit{\nameref{sec:quees}}, seguida de la motivación que ha llevado al desarrollo de este proyecto en la Sección~\ref{sec:motivacion}, \textit{\nameref{sec:motivacion}}. Finalmente se ha expuesto el estado del arte y los diferentes estudios sobre la materia realizados hasta el momento en la Sección~\ref{sec:estadodelarte}, \textit{\nameref{sec:estadodelarte}}.\\

A continuación, se procede a explicar las diferentes características del sistema, comenzando por un manual de usuario en el Capítulo~\ref{chap:guiauso}, \textit{\nameref{chap:guiauso}}, donde se expone la forma de usar la herramienta mediante la interfaz gráfica proporcionada. Esta guía abarca también el proceso necesario para instalar la aplicación en las distintas plataformas.\\

El diseño realizado para el sistema, incluyendo la especificación de los algoritmos usados tanto en el análisis de imagen como en la composición musical, se presenta en el Capítulo~\ref{chap:diseno}, \textit{\nameref{chap:diseno}}. En él, se detallan los usuarios a los que está orientada la aplicación (Sección~\ref{sec:users}, \textit{\nameref{sec:users}},), así como la a totalidad de los requisitos que detallan las funcionalidades de la aplicación final (Sección~\ref{sec:requisitos}, \textit{\nameref{sec:requisitos}}). En el resto del capítulo se expone la especificación de los algoritmos desarrollados, ya sean pertenecientes a la fase de análisis (Sección~\ref{sec:algAnalisis}, \textit{\nameref{sec:algAnalisis}}) o a la fase de composición musical (Sección~\ref{sec:algComposicion}, \textit{\nameref{sec:algComposicion}}).\\

En el siguiente Capítulo, \textit{\nameref{chap:arquitectura}}, se detalla de forma exhaustiva la implementación el sistema mediante el estándar UML. Cada sección de este capítulo hace referencia a cada uno de los módulos en los que se ha dividido la arquitectura del sistema, con la excepción de las dos primeras secciones. En ellas, se define la estructura diseñada e implementada en dos piezas claves en la realización de este proyecto: el formato de representación interna de una imagen (en la Sección~\ref{sec:representacionFiguras}, \textit{\nameref{sec:representacionFiguras}}) y el de una pieza musical (Sección~\ref{sec:repMusic}, \textit{\nameref{sec:repMusic}}; estos formatos y sus correspondientes implementaciones son usados por todos los módulos definidos a continuación. Estos módulos se ordenan en el documento por orden de importancia, siendo el primero de ellos el módulo de composición de música en la Sección~\ref{sec:modcomp}, \textit{\nameref{sec:modcomp}}, seguido del módulo de análisis de imagen (Sección~\ref{sec:modanal}, \textit{\nameref{sec:modanal}}). Finalmente se tiene en cuenta tanto el módulo que se encarga de unificar los anteriores módulos mencionados (Sección~\ref{sec:arqMuphic}, \textit{\nameref{sec:arqMuphic}}) como la interfaz gráfica desarrollada para facilitar el uso de la aplicación (Sección~\ref{sec:arqgui}, \textit{\nameref{sec:arqgui}}).\\

El contenido principal del documento finaliza en el Capítulo~\ref{chap:results}, \textit{\nameref{chap:results}}. En él, se repasa el desarrollo del proyecto y los objetivos cumplidos a lo largo del mismo, dando una visión global del sistema implementado. Posteriormente se procede a explicar los diferentes usos y mercados a los que está destinada la aplicación (Sección~\ref{sec:usos}, \textit{\nameref{sec:usos}}), así como las posibles ampliaciones que se podrían realizar sobre el sistema si se continuara su desarrollo (Sección~\ref{sec:ampliaciones}, \textit{\nameref{sec:ampliaciones}}).\\

De forma adicional, el documento presenta un anexo en el que puntualiza las tecnologías usadas en los diferentes módulos que componen la aplicación. Este es el Anexo~\ref{chap:techs}, \textit{\nameref{chap:techs}}.



