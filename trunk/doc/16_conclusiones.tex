\chapter{Conclusiones}

\section{Resultados y Ejemplares obtenidos}

\todo{Hacer}

\section{Posibles usos y aplicaciones}

\todo{Hacer}

\section{Conclusiones como tal}

\todo{QUÉ TÍTULO LE PONEMOS A ESTO?}
\todo{Re-redactar}

Nuestra motivación inicial para hacer el proyecto era la ilusión de tener un programa que fuera capaz de transmitirte musicalmente lo que percibían tus ojos, sin embargo rápidamente le vimos la posibilidad de darle usos menos frívolos a la aplicación y nos planteamos distintas posibilidades.
\newline
\\La primera de ellas fue darle un uso educativo, los niños podrían adquirir conocimientos básicos sobre la música analizando distintas estructuras generadas por la aplicación a partir de dibujos sencillos (hablamos de polígonos entremezclados) que harían ellos mismos, de esta manera se divertirían dibujando figuras, analizarían el resultado del programa de manera musical y todo esto se realizaría a través de tecnologías actuales que,  hoy en día, atraen más a las nuevas generaciones que los métodos más clásicos de enseñanza.
\newline
\\Todo esto sería posible gracias a que la música compuesta, siempre y cuando se parta de una imagen sencilla, es intuitiva y asociable a la imagen que hemos seleccionado, de esta manera cuando un alumno dibujase una figura en su mente tendría una ligera idea de como sonaría después y así jugaría entre las imágenes y la música.
\\Esta idea nos vino tras probar una aplicación (//Referencia a la aplicación en la que dibujabas lineas que sonaban//) que asocia ritmos a las líneas que dibujes.
\newline
\\La segunda fuente de motivación surgió tras hablar con un amigo compositor, cuando le comentamos nuestra idea nos dijo que estaría genial pues la música que generaríamos sería algo totalmente fresco y distinto y aunque no fuese del todo correcta serviría de base para conseguir la inspiración necesaria para empezar a componer algo desde cero, y ademas permitiría hacerlo de manera entretenida pues sería  a raíz de coger imágenes que te gustasen y meterlas en la aplicación a ver si el resultado evoca algo parecido a lo que te imaginabas.

\todo{Conclusiones como tal}
