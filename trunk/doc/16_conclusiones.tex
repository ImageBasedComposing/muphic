\chapter{Conclusiones}

%\section{Resultados y Ejemplares obtenidos}

%\todo{Hacer}

%\section{Posibles usos y aplicaciones}

%\todo{Hacer}

\section{Conclusiones}

%\todo{QUÉ TÍTULO LE PONEMOS A ESTO?}
\todo{Re-redactar, anda muy verde digo amarillo}

\begin{itemize}
	
	\item idea inicial
	\item Necesidad de comprender bien la sinestesia
	\item Las posibles soluciones a los problemas de análisis y composición
	\item Posibles escenarios y usos de la aplicación

\end{itemize}

Nuestra motivación inicial para hacer el proyecto era construir una herramienta capaz de transmitir musicalmente lo que se percibe de una imagen. Pero tras lo estudiado se ha visto que la mejor aproximación para conseguirlo es a través de la comprensión de la sinestesia. Gracias a las diferentes experiencias documentadas y las investigaciones se ha podido encontrar una fuente abundante de información.\\

El proceso de conversión entre la imagen y la pieza musical se ha dividido en dos etapas secuenciadas, primero analizar una imagen y segundo la composición musical, que se puede asemejar a la etapa de estimulación visual y la etapa de estimulación auditiva. Esto nos da la posibilidad de cambiar o configurar las etapas de forma independiente para poder experimentar las diferentes teorías que puedan surgir sobre la sinestesia.\\

Esta aplicación debería satisfacer la posibilidad de ser testeada con personas sinestésicas que pudieran reconocer parcial o totalmente la similitud entre la imagen y la música. Con las diversas pruebas realizadas durante el desarrollo se ha determinado que las posibilidades de identificar la música creada a partir de las imágenes es limitada: sólo en las imágenes simples con poca información se puede anticipar (de forma aproximada) la música generada.\\

Las posibilidades que nos permite la aplicación son diversas. Tras enseñar la herramienta a un compositor, éste nos sugirió que perfectamente podría tener la utilidad de fuente de inspiración a compositores que buscan componer nuevas obras. Otra posibilidad estaría relacionada con la educación. Tanto los niños pequeños como niños descapacitados podrían expresarse y aprender a través de esta herramienta la posibilidad de crear música de forma sencilla y a la vez divertida. Además existe la posibilidad de compartir diferentes imágenes con sus composiciones.
Dado que se une la percepción visual con la musical, es una forma más de realidad aumentada que si se llega a popularizar, sería posible que aparezca una nueva tendencia de obras artísticas en diversos medios hechas expresamente para ser interpretadas como música.\\

\color{red} Lo antiguo:
Nuestra motivación inicial para hacer el proyecto era la ilusión de tener un programa que fuera capaz de transmitirte musicalmente lo que percibían tus ojos, sin embargo rápidamente le vimos la posibilidad de darle usos menos frívolos a la aplicación y nos planteamos distintas posibilidades.\\

La primera de ellas fue darle un uso educativo, los niños podrían adquirir conocimientos básicos sobre la música analizando distintas estructuras generadas por la aplicación a partir de dibujos sencillos (hablamos de polígonos entremezclados) que harían ellos mismos, de esta manera se divertirían dibujando figuras, analizarían el resultado del programa de manera musical y todo esto se realizaría a través de tecnologías actuales que,  hoy en día, atraen más a las nuevas generaciones que los métodos más clásicos de enseñanza.
\newline
\\Todo esto sería posible gracias a que la música compuesta, siempre y cuando se parta de una imagen sencilla, es intuitiva y asociable a la imagen que hemos seleccionado, de esta manera cuando un alumno dibujase una figura en su mente tendría una ligera idea de como sonaría después y así jugaría entre las imágenes y la música.
\\Esta idea nos vino tras probar una aplicación (//Referencia a la aplicación en la que dibujabas lineas que sonaban//) que asocia ritmos a las líneas que dibujes.
\newline
\\La segunda fuente de motivación surgió tras hablar con un amigo compositor, cuando le comentamos nuestra idea nos dijo que estaría genial pues la música que generaríamos sería algo totalmente fresco y distinto y aunque no fuese del todo correcta serviría de base para conseguir la inspiración necesaria para empezar a componer algo desde cero, y ademas permitiría hacerlo de manera entretenida pues sería  a raíz de coger imágenes que te gustasen y meterlas en la aplicación a ver si el resultado evoca algo parecido a lo que te imaginabas.
\color{black}