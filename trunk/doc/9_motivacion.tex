\subsection{Motivación}

Nuestra motivación inicial para hacer el proyecto era la ilusión de tener un programa que fuera capaz de transmitirte musicalmente lo que percibían tus ojos, sin embargo rapidamente le vimos la posibilidad de darle usos menos frívolos a la aplicación y nos planteamos distintas posibilidades.
\newline
\\La primera de ellas fue darle un uso educativo, los niños podrían adquirir conocimientos básicos sobre la música analizando distintas estructuras generadas por la aplicación a partir de dibujos sencillos (hablamos de poligonos entremezclados) que harían ellos mismos, de esta manera se divertirían dibujando figuras, analizarían el resultado del programa de manera musical y todo esto se realizaría a traves de tecnologías actuales que,  hoy en día, atraen más a las nuevas generaciones que los métodos más clásicos de enseñanza.
\newline
\\Todo esto sería posible gracias a que la música compuesta, siempre y cuando se parta de una imagen sencilla, es intuible y asociable a la imagen que hemos seleccionado, de esta manera cuando un alumno dibujase una figura en su mente tendría una ligera idea de como sonaría despues y así jugaría entre las imagenes y la musica.
\\Esta idea nos vino tras probar una aplicacion (//Referencia a la aplicacion en la que dibujabas lineas que sonaban//) que asocia ritmos a las líneas que dibujes.
\newline
\\La segunda fuente de motivación surgió tras hablar con un amigo compositor, cuando le comentamos nuestra idea nos dijo que estaría genial pues la música que generaríamos sería algo totalmente fresco y distinto y aunque no fuese del todo correcta serviría de base para conseguir la inspiración necesaría para empezar a componer algo desde cero, y ademas permitiría hacerlo de manera entretenida pues sería  a raíz de coger imagenes que te gustasen y meterlas en la aplicación a ver si el resultado evoca algo parecido a lo que te imaginabas.
\newline
\\Otra de las ideas con las que partimos desde el principio fue hacer la aplicación para movil, pues nos atraía el trabajar sobre una serie de dispositivos que no conocíamos, desgraciadamente tras los primeros meses de desarrollo debido a las librerías que utilizabamos vimos que la complicación que supondría portar la aplicación a movil sería mayor de lo planeada y por lo tanto a lo largo de un año lectivo no nos sería posible desarrollarlo.
\\También nos decantamos por este proyecto entre las opciones que barajabamos porque nos interesaban tanto la composición músical como el análisis e interpretación de imagenes, y gracias al proyecto pudimos ver como estaban actualmente ambos campos y acercarnos un poco a cada uno de ellos.