\section{Motivación}

\todo{Ampliar}

\todo{
razón 1. interés musical de los partipantes
\\razón 2. interés en el tratamiento de imágenes.
\\razón 3. móviles

surge la idea de tratar la sinestesia y relacionar la musica con imagen, y los alcances que puede tener. <- Profundizar, mucho...\\
}

Desde el principio la motivación principal del proyecto era la música, tanto por el interés de sus parcipantes en ese campo (y en el de la generación automática) como por las condiciones exigidas por el tutor.

La segunda motivación de este proyecto fue el interés por el tratamiento de imagenes. Dada toda la tecnología actual que trabaja con programas de captura de video o análisis de imagenes, a todos los participantes les resultaba de interés ampliar sus conocimientos sobre esos campos. Esto fue lo que llevó la ídea inicial de composición musical a ser algo mixto donde esa composición se basaría en los datos obtenidos a partir de una imágen.
\\Inicialmente la intención era que la música compuesta transmitiese las mismas sensaciones que la imagen, sin embargo, tras indagar y encontrar referencias a la sinestesia, la idea derivó a profundizar en este campo en el ámbito que concierne el proyecto, ya que estaba poco estudiado y daba una base mas científica sobre la que trabajar.\\

Otra de las ideas con la que se partió desde el principio era enfocar la aplicación a dispositivos moviles, de esta manera se podría profundizar en otro campo tecnológico en crecimiento hoy en día. Desgraciadamente, tras los primeros meses de desarrollo debido a las librerías utilizadas
se determinó que la complicación que tendría portar la aplicación a movil sería demasiado costosa en cuanto a tiempo se refiere, por ello se acabó desechando esta posibilidad.\\


\todo{Repasar, hacerlo más formal, hacer mención al tutor en lo referente a la sinestesia? (no se como ponerlo sin caer en el tono nosotros)}