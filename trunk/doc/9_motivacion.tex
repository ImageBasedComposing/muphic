\section{Motivación}
\label{sec:motivacion}

\torev{PENDIENTE DE PRIMERA REVISIÓN}\\

La principal motivación de este proyecto nace, por supuesto, del interés de los participantes en la música y su aplicación en el área de la computación. Aunque existe una lista interminable de aplicaciones orientadas a esa relación música-informática (como puede ser SunVox \cite{SunVox} en el ámbito de los sintetizadores musicales, o la famosa aplicación ToneMatrix \cite{toneMatrix}, por citar dos ejemplos), el interés de este proyecto parte de un área en particular: la composición algorítmica.\\

El campo de la composición algorítmica consta de muchos estudios y trabajos realizados sobre la materia. Sin embargo, gan parte de ellas caen dentro de dos casos: o bien se basan en el acoplamiento y unión de diferentes piezas previamente compuestas aplicándoles ciertas modificaciones (consiguiendo resultados auditivamente agradables, pero en ningún momento ``nuevos''), o bien busca una creación completa de la melodía. Como ya se comentó en la sección anterior, es este último campo el que motiva el desarrollo de este proyecto. Se busca por tanto la composición genuína de piezas musicales, útil como fuente de inspiración para usuarios compositores o la generación de música de ambiente.\\

Dentro de la composición algorítmica, el interés de los intregrantes del proyecto se centra sobre todo en dos aspectos fundamentales:

\begin{itemize}

	\item De todas las formas posibles de generación de música algoritmica existentes, se tiene especial interés en una generación determinista. Esto es, en vez de partir de algortimos genéticos o cualquier otro tipo de diseño basado en un entrada aleatoria, se desea obtener una pieza musical que suponga la representación de un objeto constante y perteneciente a conexto no auditivo. Es esta búsqueda la que lleva a plantearse el usar imagenes como entradas a estos algoritmos.
	
	\item Además, dado que se tiene una entrada gráfica al algoritmo, la interpretación de la misma no esté sujeta a concepciones, ya sean culturales o personales, de nosotros que diseñamos los algoritmos de composición. Buscando cumplir este objetivo es donde nos encuentramos con la sinestesia, fenómeno que relaciona diferentes sentidos que además está siendo fuertemente estudiada por la rama de la psicología.
	
\end{itemize}

Se relacionan así dos elementos que incitan gran interés en la comunidad científica y que, si bien han sido estudiados por separado (como bien se aprecia en la siguiente sección), juntos componen un objeto de estudio apenas observado. Es la motivación de este proyecto el estudiar y experimentar en este ámbito, con el objetivo de expandir su trasfondo académico y observar las posibilidades que ofrece.\\

Cabe destacar también la inclinación a crear una aplicación de esta índole para dispositivos móviles. Una versión simple y accesible de este sistema puede ser de gran interés en este mercado, ya que las entradas gráficas se pueden obtener con gran facilidad gracias a las cámaras integradas en la mayoría de las plataformas portátiles. También se facilita enormemente el proceso de testeo de los diferentes resultados permitiendo su rápido progreso.