\section{Motivación}


La principal motivación de este proyecto nace, por supuesto, del interés de los participantes en la música y su aplicación en el área de la computación. Aunque existe una lista interminable de aplicaciones orientadas a esa relación música-informática (como puede ser SunVox (ver \cite{SunVox}) en el ámbito de los sintetizadores musicales, o la famosa aplicación ToneMatrix (ver \cite{toneMatrix}), por citar dos ejemplos), el interés de este proyecto parte de un área en particular: la composición algorítmica.\\

El campo de la composición algorítmica consta de muchos estudios y trabajos realizados sobre la materia que no merece la pena precisar en el presente documento. Sin embargo, gan parte de ells caen dentro de dos casos: o bien se basan en el acoplamiento y unión de diferentes piezas previamente compuestas aplicándoles ciertas modificaciones (consiguiendo resultados auditivamente ricos y agradables, pero en ningún momento ``nuevos''), o bien busca una creación completa de la melodía. Como bien se comenta en la sección anterior, es este último campo el que motiva el desarrollo de este proyecto, a pesar de que sus resultados a lo largo de diferentes estudios y pruebas no hayan llegado a ser tan agradables. Se busca por tanto la composición genuína de piezas musicales, útil como fuente de inspiración para usuarios compositores o la generación de música de ambiente.\\

Dentro de la composición algorítmica, el interés de los participantes se centra sobre todo en dos aspectos fundamentales:

\begin{itemize}

	\item De todas las formas posibles de generación de música algoritmica existentes, se tiene especial interés en una generación determinista. Esto es, en vez de partir de algortimos genéticos o cualquier otro tipo de diseño basado en un entrada aleatoria, se desea obtener una pieza musical que suponga la representación de un objeto constante y perteneciente a conexto no auditivo. Es esta búsqueda la que lleva a plantearse el usar imagenes como entradas a estos algoritmos.
	
	\item No sólo se busca la generación total y determinista de una pieza musical, sino que además esta pieza no esté sujeta a concepciones, ya sean culturales o personales, de aquellos que diseñen los algoritmos de composición. Es en esta búsqueda donde se encuentra la sinestesia, común en la mayor parte de los cerebros humanos que la experimentan (debido a su propia naturaleza), y fuertemente estudiada por la rama de la psicología.
	
\end{itemize}

Se relacionan así dos elementos que incitan gran interés en la comunidad científica y que, si bien han sido estudiados por separado (como bien se aprecia en la siguiente sección), juntos componen un objeto de estudio apenas observado. Es la motivación de este proyecto el estudiar y experimentar en este ámbito, con el objetivo de expandir su trasfondo académico y observar las posibilidades que ofrece.\\

Cabe destacar también la inclinación a crear una aplicación de esta índole para dispositivos móviles. Una versión simple y accesible de este sistema puede ser de gran interés en este mercado, ya que las entradas gráficas se pueden obtener con gran facilidad gracias a las cámaras integradas en la gran mayoría de las plataformas portátiles, facilitando enormemente el proceso de testeo de los diferentes resultados que pueden surgir en un sistema perteneciente a un ámbito tan poco estudiado, y permitiendo su rápido progreso.