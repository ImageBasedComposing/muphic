\section{Motivación}

\todo{  No se corrige, \\
Reescribir
}

Desde el principio la motivación principal del proyecto era la música, tanto por el interés de sus participantes en ese campo (y en el de la generación automática) como por las condiciones exigidas por el tutor.

La segunda motivación de este proyecto fue el interés por el tratamiento de imágenes. Dada toda la tecnología actual que trabaja con programas de captura de vídeo o análisis de imágenes, a todos los participantes les resultaba de interés ampliar sus conocimientos sobre esos campos. Esto fue lo que llevó la idea inicial de composición musical a ser algo mixto donde esa composición se basaría en los datos obtenidos a partir de una imagen.
\\Inicialmente la intención era que la música compuesta transmitiese las mismas sensaciones que la imagen, sin embargo, tras indagar y encontrar referencias a la sinestesia, la idea derivó a profundizar en este campo en el ámbito que concierne el proyecto, ya que estaba poco estudiado y daba una base mas científica sobre la que trabajar.\\

Otra de las ideas con la que se partió desde el principio era enfocar la aplicación a dispositivos móviles, de esta manera se podría profundizar en otro campo tecnológico en crecimiento hoy en día. Desgraciadamente, tras los primeros meses de desarrollo debido a las librerías utilizadas
se determinó que la complicación que tendría portar la aplicación a móvil sería demasiado costosa en cuanto a tiempo se refiere, por ello se acabó desechando esta posibilidad.\\