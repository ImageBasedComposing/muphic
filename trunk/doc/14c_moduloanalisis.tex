\section{Módulo de análisis}

\todo{referenciar figuras,	analizer, uso de opencv como paquete}


El módulo de análisis, como ya se ha comentado, es el encargado de, dada una imagen dada y una configuración de entrada, producir un archivo XML con la lista de polígonos que componen la imagen. Esta lista debe componer una nueva imagen lo más fiel posible a la imagen de entrada, teniendo en cuenta los ajustes especificados en el archivo de configuración.\\

Este módulo hace un gran uso de la librería OpenCV, de la cual se ayuda tanto para tratar imágenes como para componer formas y polígonos.\\

La estructura general del módulo es:\\

		\begin{figure}[htbp]
		\centering
		\includegraphics[scale=0.47]{graphics/todo.png}
		\caption{Vista general del módulo}
		\label{fig:diagramageneralPHIC}
		\end{figure}
		
\todo{partir de esto y seguir}		