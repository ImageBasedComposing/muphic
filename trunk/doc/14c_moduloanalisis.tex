\section{Módulo de análisis}

\todo{referenciar figuras,	analizer, uso de opencv como paquete}


El módulo de análisis, como ya se ha comentado, es el encargado de, dada una imagen dada y una configuración de entrada, producir un archivo XML con la lista de polígonos que componen la imagen. Esta lista debe componer una nueva imagen lo más fiel posible a la imagen de entrada, teniendo en cuenta los ajustes especificados en el archivo de configuración.\\

Este módulo hace un gran uso de la librería externa OpenCV, de la cual se ayuda tanto para tratar imágenes como para componer formas y polígonos.

\subsection{Vista de implementación}

La estructura general del módulo se ve en la Figura~\ref{fig:diagramaclasesPHIC}.\\

		\begin{figure}[htbp]
		\centering
		\includegraphics[scale=0.6]{graphics/diagramaclasesPHIC.png}
		\caption{Vista de las clases del módulo de anális}
		\label{fig:diagramaclasesPHIC}
		\end{figure}
		
	\todo{este diagrama contiene el nombre de ciertas funcionas (las que aparecen en el diagrame de flujo)}
		
En ella, se pueden observar las siguientes clases:\\

\begin{itemize}

	\item \textbf{Phic:} Se trata de la clase que controla la completa ejecución de este módulo. Se ocupa de coordinar el resto de clases para que realicen las operaciones requeridas para completar el análisis.
	
	\item \textbf{Analizer:} realiza todas las operaciones de análisis y se comunica con la librería OpenCV.
	
	\item \textbf{UsrConf:} como se explica en posteriores secciones, es una clase común que se ocupa de la lectura de los archivos de configuración.
	
	\item \textbf{TinyXML:} Se trata de un paquete externo de código libre que permite generar y manipular archivos XML.
	
\end{itemize}	
			
El flujo de ejecución de un análisis cualquiera realizado con este módulo se ve en la Figura~\ref{fig:diagramaflujoPHIC}.\\

		\begin{figure}[htbp]
		\centering
		\includegraphics[scale=0.47]{graphics/todo.png}
		\caption{Flujo de ejecución del módulo de anális}
		\label{fig:diagramaflujoPHIC}
		\end{figure}
		
\todo{resumen pequeño que se hace mucho mas rapido con el diagrama ya hecho}
		
\subsection{Vista de despliegue}
\todo{puede que sobre porque es obvio y UBERrepetitivo, pero es para que se vea que opencv está como dll ahí}
		
Por último, la vista final del módulo como paquetes se ve en la Figura~\ref{fig:diagramapaquetesPHIC}.\\

		\begin{figure}[htbp]
		\centering
		\includegraphics[scale=0.47]{graphics/todo.png}
		\caption{Diagrama de paquetes del módulo de anális}
		\label{fig:diagramapaquetesPHIC}
		\end{figure}
		
El sistema se basa en la librería externa OpenCV para relizar sus funciones.