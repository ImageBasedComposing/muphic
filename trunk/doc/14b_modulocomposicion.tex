\subsection{Módulo de composición}

\todo{figuras referneciar, musica referenciar,	composers, midizators, etc}

\subsubsection{Figuras Musicales}
\todo{colocar esto donde corresponda, y... ¿hacerlo menos caca?}

el módulo de composición trata las figuras según sus necesidades. Necesita saber la relevancia de cada figura además de todos los datos que se encuentran dentro de ``Figures'', para ello hereda de las clases ``Figures'' y ``Figure'' mencionadas anteriormente \todo{referencia} clases nuevas llamadas ``FiguresMusic'' y ``FigureMusic''.

\todo{Imagen mostrando la herencia?}

A la clase ``FigureMusic'' se le añade la siguiente funcionalidad:
\begin{itemize}
\item{calcularVistosidad}: Se calcula la relevancia de una figura dentro de una imagen con los siguientes valores: cantidad de rojo de la figura, cantidad de verde de la figura, cantidad de azul de la figura, área de la figura y distancia al centro de la imagen. Cada característica tiene su propio peso. La fórmula que relaciona todas las características viene en \todo{referencia a la fórmula de la parte de algorítmia}. Es necesario este valor para los compositores y para poder clasificar las figuras.
\item{compare}: Compara dos figuras según el valor de la vistosidad. Si dos figuras tienen misma vistosidad entonces tienen misma relevancia dentro de la imagen. Se usa para poder ordenar las figuras.
\end{itemize}

A ``FiguresMusic'' se le añaden los elementos necesarios para poder considerar y utilizar la nueva funcionalidad añadida a ``FigureMusic''
\begin{itemize}
\item{calcuteVisibility}: Primero calcula la vistosidad de cada figura si es que no se ha calculado ya, tomando los valores. Después se normalizan esos valores teniendo en cuenta el total de todas las figuras y la media. Se usa para poder clasificar todas las figuras.
\item{sortMusicFigures}: Dado una lista de figuras, te devuelve la lista ordenada según la vistosidad de las figuras.
\end{itemize}