\subsection{Estudio sobre Sinestesia}
\label{subsubsec:estudioSinestesia}

Esquema:
\begin{itemize}
\item introducción sinestesia color-música
\item los primeros que asociaron color con sonido
\item una aproximación a través de la psicología
\item experimentos y cosas hechas sobre la sinestesia (sin técnicas de composición)
\end{itemize}

Dentro de la sinestesia, la parte que es importante en este proyecto es la llamada sinestesia musical. Esta consiste en mezclar la experiencia auditiva con la visual. Se ha estudiado este tipo de sinestesia porque cabe pensar que sería una fuente de información razonable para poder empezar a realizar la conversión de imagen a música.
Según el neurocientífico David Eagleman las personas de forma innata tienen la tendencia de conectar sentidos, entre otros visual y auditivo (sonidos con formas y sonidos con colores) (\emph{(VideoRedesFliparColores)}).\\

Varios artistas y científicos en el tiempo han expresado su capacidad sinestésica a través de sus obras o documentando y experimentando con este fenómeno. Desde la antigua Grecia se intentaba encontrar esa equivalencia entre los colores y los sonidos. Aristóteles en su ensayo (De Sensu et Sensato \cite{DeSensuEtSensato}, 439b30) realiza una descripción de los colores comparándolos directamente con la armonía presente en la música. Siguiendo esa línea científica-filosófica Newton asignó un color a cada nota de la escala diatónica siguiendo los colores de su prisma de C (Do) rojo a B (Si) violeta.\\

Otra aproximación a la solución que normalmente siguen los artistas es a través de la experiencia con el fenómeno de la sinestésia. Encontramos entre ellos a Scriabin, compositor, quien hizo una asociación entre los colores y los acordes que aparecen en música usando el círculo de quintas. Kandinsky, otro artista como Scriabin pero pintor, partiendo de la pintura ha investigado y defendido fervientemente la posibilidad de asociar la música a la pintura y su relación directa (\cite{ConcerningSpiritualArt}). Relaciona aspectos como por ejemplo que la escala de intensidad de los sonidos estaba relacionada con la intensidad de los trazos de la pintura, también distingue timbres de instrumentos por colores asignando a cada familia colores parecidos.\\

Las investigaciones llevadas a cabo por el psicólogo Köhler en 1929 (\cite{GestaltPsychology}) y las hechas recientemente demuestran que tenemos una conexión profunda entre los sonidos y formas, además de los colores. Marks, en The Unity of the Senses (\cite{TheUnityOfTheSenses}) encuentra una asociación entre los tiempos en música y las formas, que cuanto más angular e irregular es una figura el tempo o las notas son más rápidas.También declara sobre los colores que \quote{\emph{``Por desgracia, al final uno descubre que no hay una asociación sinestésica entre notas musicales y colores que prevalezca ante las demás''}}.\\

Después de ver la aproximación a través de la experiencia y a través de la lógica o fundamentación matemática queda claro que cada persona sinestésica es única y puede haber grandes diferencias entre unas experiencias sinestésicas y otras. Es por tanto importante poder encontrar una asociación aceptable y ponerla en práctica a la hora de convertir las imágenes en música.
