\subsection{Estudio sobre Sinestesia}
\label{subsubsec:estudioSinestesia}

\torev{Última revisión realizada: 5-06-2012}\\

\color{blue} Existen varios tipos de sinestesia, de entre todos ellos este proyecto se centra en la llamada sinestesia musical. Esta consiste en mezclar la experiencia auditiva con la visual. Se estudia este tipo de sinestesia por ser una fuente razonable de información a la hora de encontrar un algoritmo para pasar de imágenes a música \color{black}
\\Según el neurocientífico David Eagleman, las personas de forma innata tienen la tendencia de conectar sentidos, entre otros visual y auditivo (sonidos con formas y sonidos con colores, \cite{VideoRedesFliparColores}).\\

Varios artistas y científicos a lo largo de los años han expresado su capacidad sinestésica a través de sus obras o documentando y experimentando este fenómeno. Desde la antigua Grecia se intentaba encontrar una equivalencia entre los colores y los sonidos, Aristóteles en su ensayo (\emph{De Sensu et Sensato} \cite{DeSensuEtSensato}, 439b30) realiza una descripción de los colores comparándolos directamente con la armonía presente en la música. \color{blue}Otra teoría dentro de la vertiente científico-filosófica la desarrolló Newton que asignó un color a cada nota de la escala diatónica (siete notas), siguiendo los colores obtenidos a partir de su prisma, que también eran siete, de tal manera que le asignó al C (Do) el color rojo (primer color del espectro visible) hasta llegar a B (Si) que se correspondería con violeta (último color). (\cite{OpticksNewton}).\\ \color{black}

Otra aproximación a la relación música-imagen que normalmente siguen los artistas es a través de la experiencia con el fenómeno de la sinestesia. Encontramos entre ellos al compositor Scriabin, quien hizo una asociación entre los colores y los acordes que aparecen en música que queda reflejada en su círculo de quintas (\cite{ScriabinQuintasColor}). Kandinsky, un pintor ruso, partiendo de la pintura ha investigado y defendido fervientemente la posibilidad de asociar la música a la pintura (\cite{ConcerningSpiritualArt}). \color{blue} Resalta aspectos como asociar la escala de intensidad de los sonidos con la intensidad de los trazos de la pintura, también distingue entre distintos timbres de instrumentos según los colores, asignando a cada familia de instrumentos colores diferentes.\\ \color{black}

Las investigaciones llevadas a cabo por el psicólogo Köhler en 1929 (\cite{GestaltPsychology}) y las hechas recientemente demuestran que tenemos una conexión profunda entre los sonidos y las formas además de los colores. Marks, en \emph{The Unity of the Senses} (\cite{TheUnityOfTheSenses}) encuentra una asociación entre los tiempos en música y las formas, tal que cuanto más angular e irregular es una figura \color{blue} más rápidas deben ser las notas o el tempo. \color{black} También declara sobre los colores que:
\begin{quote}
\emph{``Por desgracia, al final uno descubre que no hay una asociación sinestésica entre notas musicales y colores que prevalezca ante las demás''}.\\
\end{quote}

\color{blue} 
Después de ver la aproximación a través de la experiencia (con los trabajos de Scriabin o Kardinsky ya mencionados), de la lógica o las matemáticas-físicas (como los propuestos por Aristótles, Newton o Köhler) queda claro que la aptitud sinestésica de cada persona es única y puede haber grandes diferencias entre las experiencias sinestésicas de diferentes individuos. Es por tanto importante poder encontrar asociaciones aceptables y ponerlas en práctica para comprobar los distintos resultados. Para poder probar las diferentes correspondencias entre música e imagen se ha habilitado en la herramienta el poder elegir diferentes asociaciones entre las citadas en esta sección y otros autores. \color{black}