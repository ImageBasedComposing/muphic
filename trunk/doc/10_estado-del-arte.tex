\section{Estado del Arte}
\label{sec:estadodelarte}

\subsection{Estudio sobre Sinestesia}
\label{subsubsec:estudioSinestesia}

\todo{
Esquema:
\begin{itemize}
\item introducción sinestesia color-música
\item los primeros que asociaron color con sonido
\item una aproximación a través de la psicología
\item experimentos y cosas hechas sobre la sinestesia (sin técnicas de composición)
\end{itemize}
}


Dentro de la sinestesia, la parte que es importante en este proyecto es la llamada sinestesia musical. Esta consiste en mezclar la experiencia auditiva con la visual. Se ha estudiado este tipo de sinestesia porque cabe pensar que sería una fuente de información razonable para poder empezar a realizar la conversión de imagen a música.
Según el neurocientífico David Eagleman, las personas de forma innata tienen la tendencia de conectar sentidos, entre otros visual y auditivo (sonidos con formas y sonidos con colores, como se ve en \cite{VideoRedesFliparColores}).\\

Varios artistas y científicos en el tiempo han expresado su capacidad sinestésica a través de sus obras o documentando y experimentando con este fenómeno. Desde la antigua Grecia se intentaba encontrar esa equivalencia entre los colores y los sonidos. Aristóteles en su ensayo (De Sensu et Sensato \cite{DeSensuEtSensato}, 439b30) realiza una descripción de los colores comparándolos directamente con la armonía presente en la música. Siguiendo esa línea científica-filosófica, Newton asignó un color a cada nota de la escala diatónica siguiendo los colores de su prisma de C (Do) rojo a B (Si) violeta.\\

Otra aproximación a la relación música-imagen que normalmente siguen los artistas es a través de la experiencia con el fenómeno de la sinestesia. Encontramos entre ellos al compositor Scriabin, quien hizo una asociación entre los colores y los acordes que aparecen en música usando el círculo de quintas. Kandinsky, otro artista como Scriabin pero pintor, partiendo de la pintura ha investigado y defendido fervientemente la posibilidad de asociar la música a la pintura y su relación directa (\cite{ConcerningSpiritualArt}). Resalta aspectos como por ejemplo que la escala de intensidad de los sonidos estaba relacionada con la intensidad de los trazos de la pintura, también distingue timbres de instrumentos por colores asignando a cada familia colores parecidos.\\

Las investigaciones llevadas a cabo por el psicólogo Köhler en 1929 (\cite{GestaltPsychology}) y las hechas recientemente demuestran que tenemos una conexión profunda entre los sonidos y formas, además de los colores. Marks, en \emph{The Unity of the Senses} (\cite{TheUnityOfTheSenses}) encuentra una asociación entre los tiempos en música y las formas, tal que cuanto más angular e irregular es una figura el tempo o las notas son más rápidas.También declara sobre los colores que:
\begin{quote}
\emph{``Por desgracia, al final uno descubre que no hay una asociación sinestésica entre notas musicales y colores que prevalezca ante las demás''}.\\
\end{quote}

Después de ver la aproximación a través de la experiencia (con los trabajos de Aristótles, Scriabin o Kardinsky ya mencionados) a través de la lógica o fundamentación matemática (como los propuestos por Newton o Köhler), queda claro que cada persona sinestésica es única y puede haber grandes diferencias entre unas experiencias sinestésicas y otras. Es por tanto importante poder encontrar asociaciones aceptables y ponerlas en práctica para comprobar los distintos resultados.

\subsection{Composición basada en imagenes}

\torev{Última revisión realizada: 5-06-2012}

Para poder componer música a partir de datos gráficos necesitamos crear una correlación entre los elementos que tenemos en la parte gráfica y en la parte musical. Esta correspondencia no es fácil de conseguir, y a través de la historia se han desarrollado varias teorías que se han llegado a poner en práctica. Los primeros intentos de síntesis fueron los instrumentos llamados ``órganos de color'' que acompañaban al sonido con una muestra visual.

El primer instrumento lo construyó Louis Bertrand Castel (1730), se trataba de un clavecín al que se le había incorporado una pantalla y un sistema de iluminación. Cuando se apretaban las teclas se iluminaban los colores correspondientes en la pantalla (\cite{organosColor}). \color{blue} A este experimento le continuaron muchos otros, pero al ser una correspondencia muy directa entre nota y color al final se obtiene poca diversidad. Además es una correspondencia de música a la estimulación visual mientras que nuestro interés se centra en la otra dirección, de la imagen a la música.\\ \color{black}

\todo{haz aquí un itemize diciendo matemáticos entre ellos se destacan tal y cual, si dices su principal inconveniente menciona de cual o sino di en ambos su inconveniente era... etc vamos reestructurarlo de manera más legible.}
Desde hace unas décadas, gracias a los avances tecnológicos, se han investigado activamente los algoritmos automáticos de composición. Una posible clasificación de los diferentes algoritmos basada en su característica principal sería: modelos matemáticos, sistemas basados en conocimiento, gramáticas, evolutivos, sistemas con aprendizaje e híbridos (\cite{AIMethodsForComposition}). \\
\color{blue} En los algoritmos matemáticos destacamos Stochastic (Markov chains) y Chaos, su principal inconveniente es representar a alto nivel detalles más generales o abstractos de la música. En los sistemas basados en conocimiento como CHORAL o el armonizador de Pachet and Roy tienen la dificultad de representar todo el conocimiento musical necesario, que en este dominio \todo{que dominio?} es muy grande. En las gramáticas se tiene como ejemplo el proyecto EMI (Experiments in Musical Intelligence) o Steedman y su generador de música Jazz. El mayor problema que presentan las gramáticas es la rigidez de las mismas ya que la música es ambigua y con muchas excepciones a las reglas.\todo{estrcuturalos todos como este, primero en que se basan, luego el problema y por ultimo los ejemplos} Los algoritmos evolutivos su principal problema es la función de evaluación o fitness, tenemos McIntyre que usa una función de evaluación automática y la herramienta de improvisación de jazz de Biles que usa una evaluación humana. En los sistemas con aprendizaje su mayor inconveniente es que estos sistemas aprenden sólo lo básico de la música compuesta y no de los niveles más altos de composición, destacamos EBM y MUSE. Los sistemas híbridos intentan combinar lo mejor de los diferentes sistemas posibles, dentro de esta clase encontramos HARMONET que combina sistemas con aprendizaje y basados en conocimiento.
\color{black}

Pero no solo la algoritmia es variada, también podemos clasificarla de diferente forma según el tipo de música que se quiera componer: micro-composición (diseño de sonidos) y la macro-composición (combinación de sonidos ya diseñados) (\cite{AudioVisualSurvey}). \color{blue} En nuestro caso queremos separar los algoritmos de macro-composición en: basados en el fenómeno de la sinestesia y no basados en este fenómeno.\\ \color{black}

\todo{esto queda raro porque justo antes mencionas que nos interesan los algoritmos basados en la sinestesia y plantas justo un compositor no basado en ella xD}
Hay varias aproximaciones de compositores basados en imagen , algunos de ellos como Phonogramme (\cite{ImageBaseComposition} \cite{Phonogramme}) consisten en un editor gráfico que interpreta la imagen como una partitura. \color{blue} La imagen representa la relación bidimensional de tonos y duración de los sonidos, es decir, la imagen es una partitura que se lee de izquierda a derecha en el tiempo y de abajo a arriba en la altura de los tonos. 
\\El problema de este algoritmo desde el punto de vista de la sinestesia es que las imágenes son partituras y deben estar diseñadas para ser usadas para este fin (usando el editor gráfico), no acepta cualquier entrada gráfica. Esta línea de investigación no sigue o intenta imitar la sinestesia.\\ \color{black}

Otra opción interesante es la propuesta basada en el concepto de ``croma'' (o índice de cromatismo). Se subdivide la imagen en bloques (ladrillos cromáticos) los cuales tienen un índice de cromatismo, esto sirve para generar o asignar trozos de música (\cite{bricksConvertsMusic}) que pueden ser cogidos de una base de datos o ser creados por un experto (compositor). \color{blue} Al no haber una correspondencia directa entre la imagen y la música, este proyecto tampoco sigue la línea de la sinestesia.\\ \color{black}

\todo{no se si dividir los pasos en una lista o no, lés daría mas importancia de la que quizas se merecen pero quedaría mas claro}
También destacamos el trabajo realizado por Xiaoying Wu y Ze-Nian Li (\cite{ImageBaseComposition}) en el que se analiza la imagen en tres pasos. Primero: \emph{Hacer una partición} de la imagen en piezas (divisiones, trozos) más pequeñas. Segundo: realizar la \emph{Secuenciación} de esas piezas para darles un orden en el tiempo. Tercero: aplicar un \emph{Mapeado} de las piezas de imagen a notas musicales. \color{blue} 
\\Al ser una correspondencia directa entre la imagen y la música, este trabajo sigue parcialmente la sinestesia y la psicología de las formas y colores aunque este no es su propósito central.\\ \color{black}

\color{blue}
Por último destacamos el trabajo de A. Pintado (\cite{portutesis}) en el que hace una investigación de la sinestesia y la percepción de las formas para poder generar ritmos. Analizando una imagen de entrada sacando sus formas y líneas hace una relación directa con los diferentes ritmos que genera. Este trabajo se centra en la línea de la sinestesia \todo{que tipo de sinestesia? si es mas que las formas no has dicho nada sobre los colores, sino tendrías que decir por algún lado que no es el mismo tipo de sinestesia que el neustro}y por tanto ha sido una fuente de información importante. \color{black}

Todos estos algoritmos tienen en común un problema implícito puesto que dada una imagen cada persona espera una correspondencia musical diferente. Por tanto es difícil determinar el fitness o validez de cada algoritmo sabiendo además que la entrada gráfica puede tener infinidad de interpretaciones de todos sus valores disponibles.

