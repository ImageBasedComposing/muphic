\section{Estado del Arte}
\label{sec:estadodelarte}

\subsection{Estudio sobre Sinestesia}
\label{subsubsec:estudioSinestesia}

\todo{
Esquema:
\begin{itemize}
\item introducción sinestesia color-música
\item los primeros que asociaron color con sonido
\item una aproximación a través de la psicología
\item experimentos y cosas hechas sobre la sinestesia (sin técnicas de composición)
\end{itemize}
}


Dentro de la sinestesia, la parte que es importante en este proyecto es la llamada sinestesia musical. Esta consiste en mezclar la experiencia auditiva con la visual. Se ha estudiado este tipo de sinestesia porque cabe pensar que sería una fuente de información razonable para poder empezar a realizar la conversión de imagen a música.
Según el neurocientífico David Eagleman, las personas de forma innata tienen la tendencia de conectar sentidos, entre otros visual y auditivo (sonidos con formas y sonidos con colores, como se ve en \cite{VideoRedesFliparColores}).\\

Varios artistas y científicos en el tiempo han expresado su capacidad sinestésica a través de sus obras o documentando y experimentando con este fenómeno. Desde la antigua Grecia se intentaba encontrar esa equivalencia entre los colores y los sonidos. Aristóteles en su ensayo (De Sensu et Sensato \cite{DeSensuEtSensato}, 439b30) realiza una descripción de los colores comparándolos directamente con la armonía presente en la música. Siguiendo esa línea científica-filosófica, Newton asignó un color a cada nota de la escala diatónica siguiendo los colores de su prisma de C (Do) rojo a B (Si) violeta.\\

Otra aproximación a la relación música-imagen que normalmente siguen los artistas es a través de la experiencia con el fenómeno de la sinestesia. Encontramos entre ellos al compositor Scriabin, quien hizo una asociación entre los colores y los acordes que aparecen en música usando el círculo de quintas. Kandinsky, otro artista como Scriabin pero pintor, partiendo de la pintura ha investigado y defendido fervientemente la posibilidad de asociar la música a la pintura y su relación directa (\cite{ConcerningSpiritualArt}). Resalta aspectos como por ejemplo que la escala de intensidad de los sonidos estaba relacionada con la intensidad de los trazos de la pintura, también distingue timbres de instrumentos por colores asignando a cada familia colores parecidos.\\

Las investigaciones llevadas a cabo por el psicólogo Köhler en 1929 (\cite{GestaltPsychology}) y las hechas recientemente demuestran que tenemos una conexión profunda entre los sonidos y formas, además de los colores. Marks, en \emph{The Unity of the Senses} (\cite{TheUnityOfTheSenses}) encuentra una asociación entre los tiempos en música y las formas, tal que cuanto más angular e irregular es una figura el tempo o las notas son más rápidas.También declara sobre los colores que:
\begin{quote}
\emph{``Por desgracia, al final uno descubre que no hay una asociación sinestésica entre notas musicales y colores que prevalezca ante las demás''}.\\
\end{quote}

Después de ver la aproximación a través de la experiencia (con los trabajos de Aristótles, Scriabin o Kardinsky ya mencionados) a través de la lógica o fundamentación matemática (como los propuestos por Newton o Köhler), queda claro que cada persona sinestésica es única y puede haber grandes diferencias entre unas experiencias sinestésicas y otras. Es por tanto importante poder encontrar asociaciones aceptables y ponerlas en práctica para comprobar los distintos resultados.

\subsection{Composición basada en imagenes}

\gotrev{Última revisión realizada: 25-06-2012}\\

Para poder componer música a partir de datos gráficos necesitamos crear una correlación entre los elementos que tenemos en la parte gráfica y en la parte musical. Esta correspondencia no es fácil de conseguir, y a través de la historia se han desarrollado varias teorías que se han llegado a poner en práctica. Los primeros intentos de síntesis fueron los instrumentos llamados ``órganos de color'' que acompañaban al sonido con una muestra visual.\\

El primer instrumento lo construyó Louis Bertrand Castel (1730), se trataba de un clavecín al que se le había incorporado una pantalla y un sistema de iluminación. Cuando se pulsaban las teclas se iluminaban los colores correspondientes en la pantalla \cite{organosColor}. A este experimento le continuaron muchos otros que incorporaban mejoras, pero al ser una correspondencia muy directa entre nota y color, al final se obtiene poca diversidad. Además es una correspondencia de la música a la estimulación visual mientras que nuestro interés se centra en la otra dirección, de la imagen a la música.\\

Desde hace unas décadas, gracias a los avances tecnológicos, se han investigado activamente los algoritmos automáticos de composición. Una posible clasificación de los diferentes algoritmos basada en su característica principal sería: modelos matemáticos, sistemas basados en conocimiento, gramáticas, evolutivos, sistemas con aprendizaje e híbridos \cite{AIMethodsForComposition}.


\begin{itemize}
	\item Modelos matemáticos: los más usados son los procesos basados en sistemas estocásticos y cadenas de Markov. Como ejemplo significativo de estos modelos destaca Cybernetic Composer \cite{AIMusicSurvey}. Otra subcorriente son los basados en la teoría del caos \cite{ChaosTeoriaMusica}.
	\item Sistemas basados en conocimiento: dependiendo de cómo se represente el conocimiento y cómo se manipula podemos hacer diferentes clasificaciones. Como ejemplos relevantes se tienen CHORAL \cite{HistoryAlgorithmicComp} o SICOM \cite{SICOM}.
	\item Gramáticas: fueron las primeras técnicas usadas. Se suelen mezclar con técnicas probabilísticas obteniendo gramáticas indeterministas, ya que si no, puede producirse música poco variada. Destacan el proyecto EMI \cite{HistoryAlgorithmicComp} o Steedman y su generador de música Jazz \cite{AIMethodsForComposition}.
	\item Algoritmos evolutivos: se dividen en dos posibilidades según la función de evaluación. La primera es usando una función de evaluación automática. Como ejemplo se tiene McIntyre \cite{AIMethodsForComposition}. La otra posibilidad es usar una evaluación humana, que es bastante más lenta y además ambigua. La herramienta de improvisación de Jazz de Biles, GenJam \cite{GenJam}, es un ejemplo importante de este tipo.
	\item Sistemas con aprendizaje: están abiertas varias líneas de investigación según las diferentes formas del proceso de aprendizaje (adquisición de conocimiento del sistema). Una posibilidad es través de redes neuronales artificiales, un ejemplo significativo es EBM \cite{AIMethodsForComposition}. Otra manera es con aprendizaje automático (aprendizaje máquina) donde destaca el ejemplo de MUSE \cite{AIMethodsForComposition}.
	\item Híbridos: intentan combinar lo mejor que ofrecen los diferentes sistemas posibles, un ejemplo relevante es HARMONET \cite{AIMethodsForComposition} que combina sistemas con aprendizaje (redes neuronales) con sistemas basados en conocimiento.\\
\end{itemize}


Pero no solo la algoritmia es variada, también podemos clasificarla de diferente forma según el tipo de música que se quiera componer: micro-composición (diseño de sonidos) y la macro-composición (combinación de sonidos ya diseñados para la creación de una obra musical) \cite{AudioVisualSurvey}. En nuestro caso nos interesan los algoritmos de macro-composición basados en el fenómeno de la sinestesia.\\ 

Hay varias aproximaciones de compositores basados en imagen , algunos de ellos como Phonogramme \cite{ImageBaseComposition} \cite{Phonogramme} consisten en un editor gráfico que interpreta la imagen como una partitura. La imagen representa la relación bidimensional de tonos y duración de los sonidos, es decir, la imagen es una partitura que se lee de izquierda a derecha en el tiempo y de abajo a arriba en la altura de los tonos. 
\\El problema de este algoritmo desde el punto de vista de la sinestesia es que las imágenes son partituras y deben estar diseñadas para ser usadas con este fin (usando el editor gráfico), no acepta cualquier entrada gráfica. Esta línea de investigación se hace valer de la consexión psicológica que tenemos entre formas y sonidos, pero realmente no busca la sinestesia como base.\\

Otra opción es la propuesta basada en el concepto de ``croma'' (o índice de cromatismo) \cite{bricksConvertsMusic}. Se subdivide la imagen en bloques (ladrillos cromáticos) los cuales tienen un índice de cromatismo, esto sirve para generar o asignar trozos de música que pueden ser cogidos de una base de datos o ser creados por un experto (compositor). Al no haber una correspondencia directa entre la imagen y la música, este proyecto no está fundamentado en la sinestesia. Si bien recoge algunas ideas más adelante pierde la esencia de la sinestesia.\\

También destacamos el trabajo realizado por Xiaoying Wu y Ze-Nian Li \cite{ImageBaseComposition} en el que se analiza la imagen en tres pasos. Primero: hacer una \emph{Partición} de la imagen en piezas (divisiones, trozos) más pequeñas. Segundo: realizar la \emph{Secuenciación} de esas piezas para darles un orden en el tiempo. Tercero: aplicar un \emph{Mapeado} de las piezas de imagen a notas musicales.
\\Al ser una correspondencia directa entre la imagen y la música, este trabajo sigue parcialmente la sinestesia y la psicología de las formas y colores aunque este no es su propósito final.\\ 

De forma adicional, cabe mencionar dos trabajos interesantes en un ámbito menos académico.
El primero es un trabajo realizado por un equipo ruso \cite{dibujosymusica}, que permite hacer un dibujo simple (compuesto por un único trazo) y observar cómo se transforma en una melodía distinta dependiendo del trazo realizado y la herramienta con la que se ha realizado. Aunque parte de una base musical estática, a la que van ``maquillando'' de distintas formas dependiendo de la entrada gráfica, es digno de mención su facilidad de uso y la calidad de los resultados obtenidos.\\

El segundo de ellos es el realizado por el físico Lauri Gröhn \cite{rusofotos}, que aplica una serie de reglas basadas en la sinestesia para realizar composiciones algorítmicas postprocesadas. Basa el análisis en un estudio por secciones de la figura de entrada, mediante el cual va generando pequeñas porciones musicales a medida que recorre distintas regiones de la imagen divididas previamente de forma uniforme.\\

Por último destacamos el trabajo de A. Pintado \cite{portutesis} en el que hace una investigación de la sinestesia y la percepción de las formas para poder generar ritmos. Analizando una imagen de entrada obtiene sus formas y líneas los cuales traduce a ritmos que genera usando  una relación directa entre el ritmo y la inestabilidad de las figuras. Este trabajo se centra en la línea de la sinestesia, especialmente en las formas de las figuras y los ritmos musicales, dejando de lado los colores y los tonos musicales. Por tanto ha sido una buena fuente de información aunque trate sólo parcialmente nuestro objetivo.\\ 

Todos estos algoritmos tienen en común un problema implícito puesto que dada una imagen cada persona espera una correspondencia musical diferente. Por tanto es difícil determinar el fitness o validez de cada algoritmo sabiendo además que la entrada gráfica puede tener infinidad de interpretaciones de todos sus valores disponibles.

