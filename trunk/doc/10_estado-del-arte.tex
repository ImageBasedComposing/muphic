\section{Estado del Arte}
\label{sec:estadodelarte}

\subsection{Estudio sobre Sinestesia}
\label{subsubsec:estudioSinestesia}

\todo{
Esquema:
\begin{itemize}
\item introducción sinestesia color-música
\item los primeros que asociaron color con sonido
\item una aproximación a través de la psicología
\item experimentos y cosas hechas sobre la sinestesia (sin técnicas de composición)
\end{itemize}
}


Dentro de la sinestesia, la parte que es importante en este proyecto es la llamada sinestesia musical. Esta consiste en mezclar la experiencia auditiva con la visual. Se ha estudiado este tipo de sinestesia porque cabe pensar que sería una fuente de información razonable para poder empezar a realizar la conversión de imagen a música.
Según el neurocientífico David Eagleman, las personas de forma innata tienen la tendencia de conectar sentidos, entre otros visual y auditivo (sonidos con formas y sonidos con colores, como se ve en \cite{VideoRedesFliparColores}).\\

Varios artistas y científicos en el tiempo han expresado su capacidad sinestésica a través de sus obras o documentando y experimentando con este fenómeno. Desde la antigua Grecia se intentaba encontrar esa equivalencia entre los colores y los sonidos. Aristóteles en su ensayo (De Sensu et Sensato \cite{DeSensuEtSensato}, 439b30) realiza una descripción de los colores comparándolos directamente con la armonía presente en la música. Siguiendo esa línea científica-filosófica, Newton asignó un color a cada nota de la escala diatónica siguiendo los colores de su prisma de C (Do) rojo a B (Si) violeta.\\

Otra aproximación a la relación música-imagen que normalmente siguen los artistas es a través de la experiencia con el fenómeno de la sinestesia. Encontramos entre ellos al compositor Scriabin, quien hizo una asociación entre los colores y los acordes que aparecen en música usando el círculo de quintas. Kandinsky, otro artista como Scriabin pero pintor, partiendo de la pintura ha investigado y defendido fervientemente la posibilidad de asociar la música a la pintura y su relación directa (\cite{ConcerningSpiritualArt}). Resalta aspectos como por ejemplo que la escala de intensidad de los sonidos estaba relacionada con la intensidad de los trazos de la pintura, también distingue timbres de instrumentos por colores asignando a cada familia colores parecidos.\\

Las investigaciones llevadas a cabo por el psicólogo Köhler en 1929 (\cite{GestaltPsychology}) y las hechas recientemente demuestran que tenemos una conexión profunda entre los sonidos y formas, además de los colores. Marks, en \emph{The Unity of the Senses} (\cite{TheUnityOfTheSenses}) encuentra una asociación entre los tiempos en música y las formas, tal que cuanto más angular e irregular es una figura el tempo o las notas son más rápidas.También declara sobre los colores que:
\begin{quote}
\emph{``Por desgracia, al final uno descubre que no hay una asociación sinestésica entre notas musicales y colores que prevalezca ante las demás''}.\\
\end{quote}

Después de ver la aproximación a través de la experiencia (con los trabajos de Aristótles, Scriabin o Kardinsky ya mencionados) a través de la lógica o fundamentación matemática (como los propuestos por Newton o Köhler), queda claro que cada persona sinestésica es única y puede haber grandes diferencias entre unas experiencias sinestésicas y otras. Es por tanto importante poder encontrar asociaciones aceptables y ponerlas en práctica para comprobar los distintos resultados.

\subsection{Composición basada en imagenes}

\todo{ Esquema de la sección:

\begin{itemize}
\item Durante la historia los experimentos de color-música
\item Técnicas de composición más actuales
\item Pequeña conclusión o coletilla...
\end{itemize}

}

Para poder componer música a partir de datos gráficos necesitamos crear una correlación entre los elementos que tenemos en la parte gráfica y en la parte musical. Esta correspondencia no es fácil de conseguir, y a través de la historia se han desarrollado varias teorías que se han puesto en práctica. Los primeros intentos de síntesis fueron los instrumentos llamados ``órganos de color'' que mostraban acompañando al sonido una muestra visual.

El primer instrumento que se construyó fue por parte de Louis Bertrand Castel (1730) y se trataba de un clavecín al que le incorporaron una pantalla encima y un sistema de iluminación. Cuando se apretaban las teclas se iluminaban los colores correspondientes en la pantalla (\cite{organosColor}). A este experimento le continuaron muchos otros, pero al ser una correspondencia muy directa entre nota y color da poca diversidad, además es una correspondencia de música a visual.

Desde hace unas décadas, gracias a los ordenadores, se ha investigado activamente algoritmos automáticos de composición. Una posible clasificación de los diferentes algoritmos basada en su característica principal sería: modelos matemáticos, sistemas basados en conocimiento, gramáticas, evolutivos, sistemas con aprendizaje e híbridos (\cite{AIMethodsForComposition}). Pero no solo la algoritmia es variada, también podemos clasificar en diferentes forma según el tipo de música que se quiera componer: micro-composición (diseño de sonidos) y la macro-composición (combinación de sonidos ya diseñados) (\cite{AudioVisualSurvey}).

Hay varias aproximaciones de compositores basados en imagen , algunos de ellos como Phonogramme (\cite{Phonogramme} Lesbros, 1996) que es un editor gráfico, interpreta la imagen como una partitura. La imagen representa la relación bidimensional de tonos y duración de los sonidos. El problema de este algoritmo desde el punto de vista de la sinestesia es que las imágenes son partituras y deben estar diseñadas para se usadas para ese fin, no acepta cualquier entrada gráfica. 

Otra opción interesante es la propuesta basada en el concepto de ``croma'' (o índice de cromatismo). Se subdivide la imagen en bloques (ladrillos cromáticos) los cuales tienen un índice de cromatismo y sirve para generar o asignar trozos de música (\cite{bricksConvertsMusic}). Estos trozos de música pueden cogerse de una base de datos o ser creados por un experto (compositor).

Por último destacamos el trabajo realizado por Xiaoying Wu y Ze-Nian Li (\cite{ImageBaseComposition}) en el que analizan la imagen en tres pasos. Primero \emph{Hacer una partición} de la imagen en piezas (divisiones, trozos) más pequeñas. Segundo realizan la \emph{Secuenciación} de esas piezas para darles un orden en el tiempo. Por último aplican un \emph{Mapeado} de las piezas de imagen en notas musicales.

Todos estos algoritmos tienen en común un problema, que está implícito, y es que dada una imagen cada persona espera una música diferente como correspondencia. Por tanto es difícil determinar el fitness o validez de cada algoritmo sabiendo además que la entrada gráfica puede tener infinidad de valores posibles.

