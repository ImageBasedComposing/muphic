\phantomsection
\chapter*{Prefacio}
\addcontentsline{toc}{chapter}{Prefacio}

\small

Synaesthesia is a neurological phenomenon in which stimulation of one sense induces the stimulus of another, different sense. It is common for people who experience -or have experienced- this condition to identify certain images with an associated musical tune, and vice versa. This process is performed in two stages: first, the image is analyzed in order to obtain an internal representation of it, then this representation is used as an input to a synaesthetic composition system. The development of this project also includes a graphic interface, which helps the user operate the system, as well as allows him/her to handle the way the processes of analysis and composition work, to some extent. The aim of this project is to provide a multi-sensorial experience similar to synaesthesia, so that it can be used for purely artistic purposes (for example, a composer may want to use the application in order to look for image-based inspiration when it comes to writing a musical work) as well as educative purposes, as it can be used as a learning support tool with pupils in the early years that have not yet fully developed their senses.\\

\noindent\textbf{Keywords}: synaesthesia, composition, music, image, analysis\\

\vspace{0.3in}



La sinestesia es un fenómeno neurológico en el que la estimulación de un sentido induce un estímulo en otro sentido distinto. Dentro de las personas que experimentan (o han experimentado) esta condición, es muy común la identificación de imagenes asociadas a una melodía musical y viceversa. Se presenta aquí un sistema capaz de transformar una imagen dada en una pieza musical que la identifique. Este proceso se realiza en dos fases: en primer lugar se analiza la imagen hasta tener una representación interna de la misma, y posteriormente se utiliza esa representación como entrada de un sistema de composición sinestésica. El desarrollo del proyecto incluye además una interfaz gráfica que facilita al usuario la utilización del sistema, así como un control en cierta medida de los procesos de análisis y composición. El objetivo de este proyecto es proporcionar una experiencia multisensorial similar a la sinestesia, de forma que pueda ser usada tanto con fines puramente artísticos (por poner un ejemplo, un compositor puede usar la aplicación para buscar inspiración en base a una imagen a la hora de crear una obra musical) como educativos, pudiendo servir como herramienta de apoyo al aprendizaje en alumnos de temprana edad que aún estan desarrollando sus sentidos.\\


\noindent\textbf{Palabras clave}: sinestesia, composición, música, imagen, análisis\\


\normalsize