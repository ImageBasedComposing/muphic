\phantomsection
\chapter*{Prefacio}
\addcontentsline{toc}{chapter}{Prefacio}
\torev{Pendiente de primera revisión}

\small

Synaesthesia is a neurological phenomenon in which stimulation of one sense induces the stimulus of another different sense. It is common for people who experience -or have experienced- this condition to identify certain images with an associated musical piece, and vice versa. This document presents a system which is able to convert a given image into a musical piece that characterize it. This process is performed in two stages: first, the image is analyzed in order to obtain an internal representation of it, then this representation is used as an input to a synaesthesia-based composition algorithm. The development of this project also includes a graphical interface, which helps the user operate the system, as well as allows him/her to handle the way the stages of analysis and composition work, to some extent. The aim of this project is to provide a multi-sensorial experience similar to synaesthesia, so that it can be used not only for purely artistic purposes, but for research and education objectives as well.\\

\noindent\textbf{Keywords}: synaesthesia, composition, music, image, analysis.\\

\vspace{0.3in}



La sinestesia es un fenómeno neurológico en el que la estimulación de un sentido induce un estímulo en otro sentido distinto. Dentro de las personas que experimentan (o han experimentado) esta condición, es muy común la identificación de imágenes asociadas a una pieza musical y viceversa. Se presenta un sistema capaz de transformar una imagen dada en una pieza musical que le corresponda. Este proceso se realiza en dos fases: en primer lugar se analiza la imagen hasta tener una representación interna de la misma, y posteriormente se utiliza esa representación como entrada de un sistema de composición algorítmica basado en la sinestesia. El desarrollo del proyecto incluye además una interfaz gráfica que facilita al usuario el uso del sistema, así como un control en cierta medida de las fases de análisis y composición. El objetivo de este proyecto es proporcionar una experiencia multisensorial similar a la sinestesia, de forma que pueda ser usada no sólo con fines puramente artísticos sino también con objetivos de investigación o educativos.\\


\noindent\textbf{Palabras clave}: sinestesia, composición, música, imagen, análisis.\\


\normalsize